\documentclass[]{article}
\usepackage{lmodern}
\usepackage{amssymb,amsmath}
\usepackage{ifxetex,ifluatex}
\usepackage{fixltx2e} % provides \textsubscript
\ifnum 0\ifxetex 1\fi\ifluatex 1\fi=0 % if pdftex
  \usepackage[T1]{fontenc}
  \usepackage[utf8]{inputenc}
\else % if luatex or xelatex
  \ifxetex
    \usepackage{mathspec}
  \else
    \usepackage{fontspec}
  \fi
  \defaultfontfeatures{Ligatures=TeX,Scale=MatchLowercase}
\fi
% use upquote if available, for straight quotes in verbatim environments
\IfFileExists{upquote.sty}{\usepackage{upquote}}{}
% use microtype if available
\IfFileExists{microtype.sty}{%
\usepackage[]{microtype}
\UseMicrotypeSet[protrusion]{basicmath} % disable protrusion for tt fonts
}{}
\PassOptionsToPackage{hyphens}{url} % url is loaded by hyperref
\usepackage[unicode=true]{hyperref}
\hypersetup{
            pdftitle={Replication document for Propagating chronological uncertainty with Bayesian regression models},
            pdfauthor={W. Christopher Carleton; Dave Campbell},
            pdfborder={0 0 0},
            breaklinks=true}
\urlstyle{same}  % don't use monospace font for urls
\usepackage[margin=1in]{geometry}
\usepackage{color}
\usepackage{fancyvrb}
\newcommand{\VerbBar}{|}
\newcommand{\VERB}{\Verb[commandchars=\\\{\}]}
\DefineVerbatimEnvironment{Highlighting}{Verbatim}{commandchars=\\\{\}}
% Add ',fontsize=\small' for more characters per line
\usepackage{framed}
\definecolor{shadecolor}{RGB}{248,248,248}
\newenvironment{Shaded}{\begin{snugshade}}{\end{snugshade}}
\newcommand{\KeywordTok}[1]{\textcolor[rgb]{0.13,0.29,0.53}{\textbf{#1}}}
\newcommand{\DataTypeTok}[1]{\textcolor[rgb]{0.13,0.29,0.53}{#1}}
\newcommand{\DecValTok}[1]{\textcolor[rgb]{0.00,0.00,0.81}{#1}}
\newcommand{\BaseNTok}[1]{\textcolor[rgb]{0.00,0.00,0.81}{#1}}
\newcommand{\FloatTok}[1]{\textcolor[rgb]{0.00,0.00,0.81}{#1}}
\newcommand{\ConstantTok}[1]{\textcolor[rgb]{0.00,0.00,0.00}{#1}}
\newcommand{\CharTok}[1]{\textcolor[rgb]{0.31,0.60,0.02}{#1}}
\newcommand{\SpecialCharTok}[1]{\textcolor[rgb]{0.00,0.00,0.00}{#1}}
\newcommand{\StringTok}[1]{\textcolor[rgb]{0.31,0.60,0.02}{#1}}
\newcommand{\VerbatimStringTok}[1]{\textcolor[rgb]{0.31,0.60,0.02}{#1}}
\newcommand{\SpecialStringTok}[1]{\textcolor[rgb]{0.31,0.60,0.02}{#1}}
\newcommand{\ImportTok}[1]{#1}
\newcommand{\CommentTok}[1]{\textcolor[rgb]{0.56,0.35,0.01}{\textit{#1}}}
\newcommand{\DocumentationTok}[1]{\textcolor[rgb]{0.56,0.35,0.01}{\textbf{\textit{#1}}}}
\newcommand{\AnnotationTok}[1]{\textcolor[rgb]{0.56,0.35,0.01}{\textbf{\textit{#1}}}}
\newcommand{\CommentVarTok}[1]{\textcolor[rgb]{0.56,0.35,0.01}{\textbf{\textit{#1}}}}
\newcommand{\OtherTok}[1]{\textcolor[rgb]{0.56,0.35,0.01}{#1}}
\newcommand{\FunctionTok}[1]{\textcolor[rgb]{0.00,0.00,0.00}{#1}}
\newcommand{\VariableTok}[1]{\textcolor[rgb]{0.00,0.00,0.00}{#1}}
\newcommand{\ControlFlowTok}[1]{\textcolor[rgb]{0.13,0.29,0.53}{\textbf{#1}}}
\newcommand{\OperatorTok}[1]{\textcolor[rgb]{0.81,0.36,0.00}{\textbf{#1}}}
\newcommand{\BuiltInTok}[1]{#1}
\newcommand{\ExtensionTok}[1]{#1}
\newcommand{\PreprocessorTok}[1]{\textcolor[rgb]{0.56,0.35,0.01}{\textit{#1}}}
\newcommand{\AttributeTok}[1]{\textcolor[rgb]{0.77,0.63,0.00}{#1}}
\newcommand{\RegionMarkerTok}[1]{#1}
\newcommand{\InformationTok}[1]{\textcolor[rgb]{0.56,0.35,0.01}{\textbf{\textit{#1}}}}
\newcommand{\WarningTok}[1]{\textcolor[rgb]{0.56,0.35,0.01}{\textbf{\textit{#1}}}}
\newcommand{\AlertTok}[1]{\textcolor[rgb]{0.94,0.16,0.16}{#1}}
\newcommand{\ErrorTok}[1]{\textcolor[rgb]{0.64,0.00,0.00}{\textbf{#1}}}
\newcommand{\NormalTok}[1]{#1}
\usepackage{graphicx,grffile}
\makeatletter
\def\maxwidth{\ifdim\Gin@nat@width>\linewidth\linewidth\else\Gin@nat@width\fi}
\def\maxheight{\ifdim\Gin@nat@height>\textheight\textheight\else\Gin@nat@height\fi}
\makeatother
% Scale images if necessary, so that they will not overflow the page
% margins by default, and it is still possible to overwrite the defaults
% using explicit options in \includegraphics[width, height, ...]{}
\setkeys{Gin}{width=\maxwidth,height=\maxheight,keepaspectratio}
\IfFileExists{parskip.sty}{%
\usepackage{parskip}
}{% else
\setlength{\parindent}{0pt}
\setlength{\parskip}{6pt plus 2pt minus 1pt}
}
\setlength{\emergencystretch}{3em}  % prevent overfull lines
\providecommand{\tightlist}{%
  \setlength{\itemsep}{0pt}\setlength{\parskip}{0pt}}
\setcounter{secnumdepth}{0}
% Redefines (sub)paragraphs to behave more like sections
\ifx\paragraph\undefined\else
\let\oldparagraph\paragraph
\renewcommand{\paragraph}[1]{\oldparagraph{#1}\mbox{}}
\fi
\ifx\subparagraph\undefined\else
\let\oldsubparagraph\subparagraph
\renewcommand{\subparagraph}[1]{\oldsubparagraph{#1}\mbox{}}
\fi

% set default figure placement to htbp
\makeatletter
\def\fps@figure{htbp}
\makeatother


\title{Replication document for `Propagating chronological uncertainty with
Bayesian regression models'}
\author{W. Christopher Carleton\footnote{Extreme Events Research Group, Max
  Planck Institutes for Chemical Ecology, Science of Human History, and
  Biogeochemistry, Jena Germany} \and Dave Campbell\footnote{School of Mathematics and Statistics, Carleton
  University, Ottawa Canada}}
\date{}

\begin{document}
\maketitle

\section{Libraries}\label{libraries}

\begin{Shaded}
\begin{Highlighting}[]
\KeywordTok{library}\NormalTok{(ggplot2)}
\KeywordTok{library}\NormalTok{(ggpubr)}
\KeywordTok{library}\NormalTok{(IntCal)}
\KeywordTok{library}\NormalTok{(coda)}
\end{Highlighting}
\end{Shaded}

\section{Run this script?}\label{run-this-script}

The analyses conducted using the code below can take a long time to run.
On a desktop with 32 cores and 32G memory, the whole script required
several hours to complete. So, when originally prepared, this code was
run in chunks, outputs were saved, and then reloaded into the R
workspace as needed for plotting results. This made it possible to
disentangle code processing from minor changes to the wording or layout
of this document.

To facilitate that separation, a set of variables beginning with
\texttt{run\_it\_*} were used as toggles and passed as an option to each
R Markdown chunk below. When a \texttt{run\_it\_*} is \texttt{TRUE}, the
code in the relevant chunk runs. When false, it is assumed that certain
results (big matrices and R objects) are already in the R workspace and
or can be loaded by the relevant code chunk. The toggles include code
chunks used for plotting and each MCMC analysis has been separated. On
first running this replication script, you will need to set all of these
toggles to \texttt{TRUE} in order to fully replicate the analysis
described in the paper associated with this document.

\begin{Shaded}
\begin{Highlighting}[]
\NormalTok{run_it_simdata_pre =}\StringTok{ }\NormalTok{F}
\NormalTok{run_it_simdata_chronup =}\StringTok{ }\NormalTok{F}
\NormalTok{run_it_mcmc =}\StringTok{ }\KeywordTok{c}\NormalTok{(F,  }\CommentTok{#J=5}
\NormalTok{                F,  }\CommentTok{#J=10}
\NormalTok{                F,  }\CommentTok{#J=50}
\NormalTok{                F,  }\CommentTok{#J=100}
\NormalTok{                F)  }\CommentTok{#chronup}
\NormalTok{run_it_plot =}\StringTok{ }\NormalTok{T}
\NormalTok{run_it_simdata_big =}\StringTok{ }\NormalTok{F}
\NormalTok{run_it_mcmc_big =}\StringTok{ }\KeywordTok{c}\NormalTok{(F, }\CommentTok{#J=50}
\NormalTok{                    F) }\CommentTok{#chronup}
\NormalTok{run_it_plot_big =}\StringTok{ }\NormalTok{T}
\end{Highlighting}
\end{Shaded}

\subsection{Directory structure}\label{directory-structure}

R markdown typically runs code chunks in the directory containing the
relevant .Rmd file (this file). But, it is common practice to separate
data and source files into different subdirectories within a project
folder. This .Rmd file assumes that the chunks below are being run in
the \texttt{Src} folder within a parent project folder. It also assumes
there is another subdirectory in the parent project folder called
\texttt{Data}. All data files and output will be saved to the
\texttt{Data} folder and it is assumed that any data loaded into the
script is in that folder as well. To change this default behaviour,
alter the path below as needed (include trailing slash).

\begin{Shaded}
\begin{Highlighting}[]
\CommentTok{# relative to the directory containing this .Rmd file}
\NormalTok{datapath <-}\StringTok{ "../Data/"}
\end{Highlighting}
\end{Shaded}

\section{Simulating Count Data}\label{simulating-count-data}

\subsection{True Process}\label{true-process}

As described in the associated paper, the simulation process is based on
a Poisson distribution with a mean defined by a regression function with
one covariate. Thus, the count \(y\) at time \(t\) is a random variable
with the following conditional distribution,

\begin{align}
    y_t | \lambda_t &\sim Pois(\lambda_t), \\
    \lambda_t &= e^{\beta X_t}.
\end{align}

To produce samples from this process while simulating the effect of
chronological uncertainty, we need to set several parameters. These
include the number of desired intervals in the sequence (\(tau\)), the
total number of events in the sequence (\(n\)), a regression coefficient
(\(\beta\)), covariate sequence (\(x\)), and start and end times for the
count sequence chronology. For the sake of simplicity, we will assume
that the time dimension is measured using a Before Present (BP) system
and that the intervals correspond to years. As a result, the passage of
time in the analysis that follows corresponds with decreasing age (a
given number of years before the present).

\begin{Shaded}
\begin{Highlighting}[]
\NormalTok{nevents <-}\StringTok{ }\DecValTok{1000}
\NormalTok{tau <-}\StringTok{ }\DecValTok{1000}
\NormalTok{beta <-}\StringTok{ }\FloatTok{0.004}
\NormalTok{x <-}\StringTok{ }\DecValTok{0}\OperatorTok{:}\NormalTok{(tau }\OperatorTok{-}\StringTok{ }\DecValTok{1}\NormalTok{)}
\NormalTok{start <-}\StringTok{ }\DecValTok{5000}
\NormalTok{end <-}\StringTok{ }\NormalTok{(start }\OperatorTok{-}\StringTok{ }\NormalTok{tau) }\OperatorTok{+}\StringTok{ }\DecValTok{1}
\end{Highlighting}
\end{Shaded}

With these parameters set, we can calculate the conditional mean for the
Poisson process. We will then randomly sample \(n\) ages from
\([start, end]\) where the probability of sampling any given age in that
range corresponds to the conditional mean function. This can be
accomplished by passing the mean function evaluated at a discrete
evenly-spaced set of times to R's built-in \texttt{sample()} function.
Then, we can use the built-in \texttt{graphics::hist()} function to bin
the sampled ages into a count sequence with the help of the
\texttt{breaks} argument to define the temporal bin edges.

\begin{Shaded}
\begin{Highlighting}[]
\NormalTok{lambda <-}\StringTok{ }\KeywordTok{exp}\NormalTok{(beta }\OperatorTok{*}\StringTok{ }\NormalTok{x)}
\NormalTok{pois_process <-}\StringTok{ }\KeywordTok{data.frame}\NormalTok{(}\DataTypeTok{lambda =}\NormalTok{ lambda)}
\NormalTok{ages <-}\StringTok{ }\KeywordTok{sample}\NormalTok{(}\DataTypeTok{x =}\NormalTok{ start}\OperatorTok{:}\NormalTok{end,}
            \DataTypeTok{size =}\NormalTok{ nevents,}
            \DataTypeTok{prob =}\NormalTok{ lambda,}
            \DataTypeTok{replace =}\NormalTok{ T)}
\NormalTok{y <-}\StringTok{ }\NormalTok{graphics}\OperatorTok{::}\KeywordTok{hist}\NormalTok{(ages,}
                    \DataTypeTok{breaks =} \KeywordTok{seq}\NormalTok{(start, end }\OperatorTok{-}\StringTok{ }\DecValTok{1}\NormalTok{, }\DecValTok{-1}\NormalTok{),}
                    \DataTypeTok{include.lowest =} \OtherTok{FALSE}\NormalTok{,}
                    \DataTypeTok{plot =} \OtherTok{FALSE}\NormalTok{)}\OperatorTok{$}\NormalTok{counts}
\NormalTok{simdata <-}\StringTok{ }\KeywordTok{data.frame}\NormalTok{(}\DataTypeTok{timestamps =}\NormalTok{ start}\OperatorTok{:}\NormalTok{end,}
                    \DataTypeTok{lambda =}\NormalTok{ lambda,}
                    \DataTypeTok{y =} \KeywordTok{rev}\NormalTok{(y),}
                    \DataTypeTok{x =}\NormalTok{ x)}
\KeywordTok{save}\NormalTok{(simdata,}
    \DataTypeTok{file =} \KeywordTok{paste}\NormalTok{(datapath,}
                \StringTok{"simdata.RData"}\NormalTok{,}
                \DataTypeTok{sep =} \StringTok{""}\NormalTok{))}
\end{Highlighting}
\end{Shaded}

Just to be certain that the sample looks sensible, plot the sample along
with the underlying process (conditional Poisson mean).

\begin{Shaded}
\begin{Highlighting}[]
\KeywordTok{load}\NormalTok{(}\KeywordTok{paste}\NormalTok{(datapath,}
            \StringTok{"simdata.RData"}\NormalTok{,}
            \DataTypeTok{sep =} \StringTok{""}\NormalTok{))}
\NormalTok{p1 <-}\StringTok{ }\KeywordTok{ggplot}\NormalTok{(simdata) }\OperatorTok{+}
\StringTok{            }\KeywordTok{geom_path}\NormalTok{(}\KeywordTok{aes}\NormalTok{(}\DataTypeTok{y =}\NormalTok{ lambda, }\DataTypeTok{x =}\NormalTok{ timestamps)) }\OperatorTok{+}
\StringTok{            }\KeywordTok{theme_minimal}\NormalTok{()}
\NormalTok{p2 <-}\StringTok{ }\KeywordTok{ggplot}\NormalTok{(simdata) }\OperatorTok{+}
\StringTok{            }\KeywordTok{geom_col}\NormalTok{(}\KeywordTok{aes}\NormalTok{(}\DataTypeTok{y =}\NormalTok{ y, }\DataTypeTok{x =}\NormalTok{ timestamps)) }\OperatorTok{+}
\StringTok{            }\KeywordTok{theme_minimal}\NormalTok{()}
\KeywordTok{ggarrange}\NormalTok{(p1,}
\NormalTok{        p2,}
        \DataTypeTok{ncol =} \DecValTok{1}\NormalTok{,}
        \DataTypeTok{nrow =} \DecValTok{2}\NormalTok{,}
        \DataTypeTok{align =} \StringTok{"v"}\NormalTok{)}
\end{Highlighting}
\end{Shaded}

\includegraphics{replication_files/figure-latex/plot-simdata-1.pdf}

We can also quickly check to see whether the sample size is sufficient
to estimate \(\beta\) by running a simple Poisson regression with R's
\texttt{glm()} function.

\begin{Shaded}
\begin{Highlighting}[]
\NormalTok{glm_check <-}\StringTok{ }\KeywordTok{glm}\NormalTok{(y}\OperatorTok{~}\NormalTok{x,}
                \DataTypeTok{data =}\NormalTok{ simdata,}
                \DataTypeTok{family =} \StringTok{"poisson"}\NormalTok{)}
\KeywordTok{summary}\NormalTok{(glm_check)}
\end{Highlighting}
\end{Shaded}

\begin{verbatim}
## 
## Call:
## glm(formula = y ~ x, family = "poisson", data = simdata)
## 
## Deviance Residuals: 
##     Min       1Q   Median       3Q      Max  
## -2.8011  -0.8247  -0.4829   0.4459   3.2839  
## 
## Coefficients:
##               Estimate Std. Error z value Pr(>|z|)    
## (Intercept) -2.5528366  0.1195663  -21.35   <2e-16 ***
## x            0.0039473  0.0001506   26.22   <2e-16 ***
## ---
## Signif. codes:  0 '***' 0.001 '**' 0.01 '*' 0.05 '.' 0.1 ' ' 1
## 
## (Dispersion parameter for poisson family taken to be 1)
## 
##     Null deviance: 1930.71  on 999  degrees of freedom
## Residual deviance:  990.22  on 998  degrees of freedom
## AIC: 2163.3
## 
## Number of Fisher Scoring iterations: 5
\end{verbatim}

\subsection{Introducing Chronological
Uncertainty}\label{introducing-chronological-uncertainty}

Adding chronological uncertainty to the simulated data could be done in
many ways. But, since radiocarbon dating is probably the most commonly
used chronometric method in the Palaeo Sciences (and very popular
recently with respect to event count sequences), we will simulate
radiocarbon uncertainties.

To begin, pass the sampled ages from above to the IntCal packages's
\texttt{calBP.14C} function. The function returns plausible C14
determinations (ages in uncalibrated radiocarbon years before present)
given a set of calendar ages (the ages we sampled). These simulated C14
ages must then be calibrated and can be stored conveniently in a
chronological uncertainty matrix called \texttt{chronun\_matrix}. In
order to align the calibrated date densities (discrete estimates
provided by calibration software) for storage in the matrix, we also
need to determine the maximum span of time
(\texttt{sample\_time\_range}) covered by all the simulated sample
dates. Lastly, the range of times corresponding to that enclosing
interval can be stored in a vector separate from the
\texttt{chronun\_matrix} for use later on. The following code
accomplishes all of these steps.

\begin{Shaded}
\begin{Highlighting}[]
\NormalTok{simc14 <-}\StringTok{ }\KeywordTok{t}\NormalTok{(}\KeywordTok{sapply}\NormalTok{(ages, IntCal}\OperatorTok{::}\NormalTok{calBP}\FloatTok{.14}\NormalTok{C))}

\NormalTok{c14post <-}\StringTok{ }\KeywordTok{lapply}\NormalTok{(}\DecValTok{1}\OperatorTok{:}\NormalTok{nevents,}
                \ControlFlowTok{function}\NormalTok{(x, dates)\{}
\NormalTok{                    IntCal}\OperatorTok{::}\KeywordTok{caldist}\NormalTok{(}\DataTypeTok{age =}\NormalTok{ dates[x, }\DecValTok{1}\NormalTok{],}
                                    \DataTypeTok{error =}\NormalTok{ dates[x, }\DecValTok{2}\NormalTok{],}
                                    \DataTypeTok{BCAD =}\NormalTok{ F)\},}
                    \DataTypeTok{dates =}\NormalTok{ simc14)}
\NormalTok{sample_time_range <-}\StringTok{ }\KeywordTok{range}\NormalTok{(}
                        \KeywordTok{unlist}\NormalTok{(}
                            \KeywordTok{lapply}\NormalTok{(c14post,}
                                \ControlFlowTok{function}\NormalTok{(x)}\KeywordTok{range}\NormalTok{(x[, }\DecValTok{1}\NormalTok{])}
\NormalTok{                            )}
\NormalTok{                        )}
\NormalTok{                    )}
\NormalTok{expanded_times <-}\StringTok{ }\NormalTok{sample_time_range[}\DecValTok{2}\NormalTok{]}\OperatorTok{:}\NormalTok{sample_time_range[}\DecValTok{1}\NormalTok{]}
\NormalTok{c14post_expanded <-}\StringTok{ }\KeywordTok{lapply}\NormalTok{(c14post,}
                            \ControlFlowTok{function}\NormalTok{(caldate, times)\{}
                                    \KeywordTok{approx}\NormalTok{(caldate,}
                                        \DataTypeTok{xout =}\NormalTok{ times)}\OperatorTok{$}\NormalTok{y}
\NormalTok{                                    \}, }\DataTypeTok{times =}\NormalTok{ expanded_times)}
\NormalTok{chronun_matrix <-}\StringTok{ }\KeywordTok{do.call}\NormalTok{(cbind,c14post_expanded)}
\NormalTok{chronun_matrix[}\KeywordTok{which}\NormalTok{(}\KeywordTok{is.na}\NormalTok{(chronun_matrix))] <-}\StringTok{ }\DecValTok{0}
\end{Highlighting}
\end{Shaded}

Next, the \texttt{chronun\_matrix} is used to create an
\texttt{event\_count\_ensemble}. The ensemble is just a sample of
probable event counts given the radiocarbon dating uncertainty we just
simulated. Each member of the ensemble (one probable count sequence)
will be stored as a column in a matrix. Since we want to create a large
ensemble, two small functions will be required. One function will sample
the \texttt{chronun\_matrix} for a probable set of event ages and the
other will count those events, binning them into a probable count
sequence with the given resolution (defined by temporal bin edges passed
as the \texttt{breaks} argument to the \texttt{graphics::hist()}
function used earlier).

\begin{Shaded}
\begin{Highlighting}[]
\CommentTok{# first function - sample event times}
\NormalTok{sample_event_times <-}\StringTok{ }\ControlFlowTok{function}\NormalTok{(}\DataTypeTok{x =} \OtherTok{NULL}\NormalTok{,}
\NormalTok{                                chronun_matrix,}
\NormalTok{                                times)\{}
\NormalTok{    times_sample <-}\StringTok{ }\KeywordTok{apply}\NormalTok{(chronun_matrix,}
                            \DecValTok{2}\NormalTok{,}
                            \ControlFlowTok{function}\NormalTok{(j)}\KeywordTok{sample}\NormalTok{(times, }\DataTypeTok{size=}\DecValTok{1}\NormalTok{, }\DataTypeTok{prob=}\NormalTok{j))}
    \KeywordTok{return}\NormalTok{(times_sample)}
\NormalTok{\}}
\CommentTok{# second function - count events}
\NormalTok{count_events <-}\StringTok{ }\ControlFlowTok{function}\NormalTok{(x,}
\NormalTok{                        breaks)\{}
\NormalTok{    counts <-}\StringTok{ }\NormalTok{graphics}\OperatorTok{::}\KeywordTok{hist}\NormalTok{(x,}
                            \DataTypeTok{breaks =}\NormalTok{ breaks,}
                            \DataTypeTok{include.lowest =} \OtherTok{FALSE}\NormalTok{,}
                            \DataTypeTok{plot =} \OtherTok{FALSE}\NormalTok{)}\OperatorTok{$}\NormalTok{counts}
    \KeywordTok{return}\NormalTok{(}\KeywordTok{rev}\NormalTok{(counts))}
\NormalTok{\}}
\CommentTok{# create breaks vector}
\NormalTok{start <-}\StringTok{ }\NormalTok{expanded_times[}\DecValTok{1}\NormalTok{]}
\NormalTok{end <-}\StringTok{ }\NormalTok{expanded_times[}\KeywordTok{length}\NormalTok{(expanded_times)] }\OperatorTok{-}\StringTok{ }\DecValTok{1}
\NormalTok{breaks <-}\StringTok{ }\KeywordTok{seq}\NormalTok{(}\DataTypeTok{from =}\NormalTok{ start, }\DataTypeTok{to =}\NormalTok{ end, }\DataTypeTok{by =} \DecValTok{-1}\NormalTok{)}
\end{Highlighting}
\end{Shaded}

With the helpful functions in hand, sample 1000 probable event ages:

\begin{Shaded}
\begin{Highlighting}[]
\NormalTok{event_time_sample <-}\StringTok{ }\KeywordTok{sapply}\NormalTok{(}\DecValTok{1}\OperatorTok{:}\DecValTok{1000}\NormalTok{,}
\NormalTok{                        sample_event_times,}
                        \DataTypeTok{chronun_matrix =}\NormalTok{ chronun_matrix,}
                        \DataTypeTok{times =}\NormalTok{ expanded_times)}
\end{Highlighting}
\end{Shaded}

And, then, compile the count sequences:

\begin{Shaded}
\begin{Highlighting}[]
\NormalTok{event_count_ensemble <-}\StringTok{ }\KeywordTok{apply}\NormalTok{(event_time_sample,}
                            \DecValTok{2}\NormalTok{,}
\NormalTok{                            count_events,}
                            \DataTypeTok{breaks =}\NormalTok{ breaks)}
\KeywordTok{save}\NormalTok{(event_count_ensemble,}
    \DataTypeTok{file =} \KeywordTok{paste}\NormalTok{(datapath,}
                \StringTok{"count_ensemble_1.RData"}\NormalTok{,}
                \DataTypeTok{sep =} \StringTok{""}\NormalTok{))}

\KeywordTok{save}\NormalTok{(expanded_times,}
    \DataTypeTok{file =} \KeywordTok{paste}\NormalTok{(datapath,}
                \StringTok{"count_ensemble_1_expanded_times.RData"}\NormalTok{,}
                \DataTypeTok{sep =} \StringTok{""}\NormalTok{))}
\end{Highlighting}
\end{Shaded}

As before, it's a good idea to plot some of the sequences to make sure
they are as they should be (increasing with the passage of time in a
roughly exponential-looking way). To help keep variable names to a
reasonable length, we can refer to any event count ensemble as an
\texttt{ece} and if the events in question are radiocarbon dated we can
use \texttt{rece} (radiocarbon-dated event count ensemble).

\begin{Shaded}
\begin{Highlighting}[]
\KeywordTok{load}\NormalTok{(}\KeywordTok{paste}\NormalTok{(datapath,}
            \StringTok{"count_ensemble_1.RData"}\NormalTok{,}
            \DataTypeTok{sep =} \StringTok{""}\NormalTok{))}

\KeywordTok{load}\NormalTok{(}\KeywordTok{paste}\NormalTok{(datapath,}
            \StringTok{"count_ensemble_1_expanded_times.RData"}\NormalTok{,}
            \DataTypeTok{sep =} \StringTok{""}\NormalTok{))}

\NormalTok{rece <-}\StringTok{ }\KeywordTok{data.frame}\NormalTok{(}\DataTypeTok{timestamps =}\NormalTok{ expanded_times,}
\NormalTok{                    event_count_ensemble[,}\DecValTok{1}\OperatorTok{:}\DecValTok{4}\NormalTok{])}
\NormalTok{p1 <-}\KeywordTok{ggplot}\NormalTok{(rece) }\OperatorTok{+}
\StringTok{            }\KeywordTok{geom_col}\NormalTok{(}\KeywordTok{aes}\NormalTok{(}\DataTypeTok{y =}\NormalTok{ X1, }\DataTypeTok{x =}\NormalTok{ timestamps)) }\OperatorTok{+}
\StringTok{            }\KeywordTok{theme_minimal}\NormalTok{()}
\NormalTok{p2 <-}\StringTok{ }\KeywordTok{ggplot}\NormalTok{(rece) }\OperatorTok{+}
\StringTok{            }\KeywordTok{geom_col}\NormalTok{(}\KeywordTok{aes}\NormalTok{(}\DataTypeTok{y =}\NormalTok{ X2, }\DataTypeTok{x =}\NormalTok{ timestamps)) }\OperatorTok{+}
\StringTok{            }\KeywordTok{theme_minimal}\NormalTok{()}
\NormalTok{p3 <-}\StringTok{ }\KeywordTok{ggplot}\NormalTok{(rece) }\OperatorTok{+}
\StringTok{            }\KeywordTok{geom_col}\NormalTok{(}\KeywordTok{aes}\NormalTok{(}\DataTypeTok{y =}\NormalTok{ X3, }\DataTypeTok{x =}\NormalTok{ timestamps)) }\OperatorTok{+}
\StringTok{            }\KeywordTok{theme_minimal}\NormalTok{()}
\NormalTok{p4 <-}\StringTok{ }\KeywordTok{ggplot}\NormalTok{(rece) }\OperatorTok{+}
\StringTok{            }\KeywordTok{geom_col}\NormalTok{(}\KeywordTok{aes}\NormalTok{(}\DataTypeTok{y =}\NormalTok{ X4, }\DataTypeTok{x =}\NormalTok{ timestamps)) }\OperatorTok{+}
\StringTok{            }\KeywordTok{theme_minimal}\NormalTok{()}
\KeywordTok{ggarrange}\NormalTok{(p1,}
\NormalTok{        p2,}
\NormalTok{        p3,}
\NormalTok{        p4,}
        \DataTypeTok{ncol =} \DecValTok{2}\NormalTok{,}
        \DataTypeTok{nrow =} \DecValTok{2}\NormalTok{,}
        \DataTypeTok{align =} \StringTok{"v"}\NormalTok{)}
\end{Highlighting}
\end{Shaded}

\includegraphics{replication_files/figure-latex/plot-rece-1.pdf}

In the figure above, each plot shows one probable event count sequence
given the uncertainty in the simulated radiocarbon dates for the
corresponding events. Time in the plots proceeds right to left on the BP
timescale (easy enough to change by reversing the axes if desired).
Individually, the sequences include no information about chronological
uncertainty. In isolation a given ensemble member is effectively
assuming a certain set of ages for the events (sampled at random from
the corresponding calibrated radiocarbon-date distributions).
Collectively, though, the ensemble contains the information needed to
propagate the chronological uncertainty into any further analyses.

\section{\texorpdfstring{Simulating Count Data with
`chronup'}{Simulating Count Data with chronup}}\label{simulating-count-data-with-chronup}

All of the steps taken so far to simulate data can be repeated more
easily with an R package called
\href{https://github.com/wccarleton/chronup}{chronup}. The package is
currently under development and in a pre-release state (not yet on
CRAN). But, the code is open and available on GitHub. The package
\texttt{R::devtools} can be used to download and install the latest
version of \texttt{chronup}. To facilitate large sample of events and
very large ensembles, the package can make use of parallelization
(built-in R functions) and big matrix objects (from
\texttt{R::bigmemory}). So, it is assumed from here on that this script
is being run in an environment that supports both.

\begin{verbatim}
library(devtools)
install_github("wccarleton/chronup")
\end{verbatim}

As before, we can define some key parameters to begin. In addition to
the same ones defined above, a new parameter called \texttt{nsamples}
will determine how many probable event count sequences to create.

\begin{Shaded}
\begin{Highlighting}[]
\NormalTok{nevents <-}\StringTok{ }\DecValTok{1000}
\NormalTok{nsamples <-}\StringTok{ }\DecValTok{200000}
\NormalTok{tau <-}\StringTok{ }\DecValTok{1000}
\NormalTok{beta <-}\StringTok{ }\FloatTok{0.004}
\NormalTok{x <-}\StringTok{ }\DecValTok{0}\OperatorTok{:}\NormalTok{(tau }\OperatorTok{-}\StringTok{ }\DecValTok{1}\NormalTok{)}
\NormalTok{lambda <-}\StringTok{ }\KeywordTok{exp}\NormalTok{(beta }\OperatorTok{*}\StringTok{ }\NormalTok{x)}
\NormalTok{start <-}\StringTok{ }\DecValTok{5000}
\NormalTok{end <-}\StringTok{ }\NormalTok{(start }\OperatorTok{-}\StringTok{ }\NormalTok{tau) }\OperatorTok{+}\StringTok{ }\DecValTok{1}
\NormalTok{times <-}\StringTok{ }\NormalTok{start}\OperatorTok{:}\NormalTok{end}
\end{Highlighting}
\end{Shaded}

With these parameters set, one function call to
\texttt{chronup::simulate\_event\_counts} will produce 200,000 probable
event count sequences assuming radiocarbon-dated uncertainties:

\begin{Shaded}
\begin{Highlighting}[]
\NormalTok{simdata2 <-}\StringTok{ }\NormalTok{chronup}\OperatorTok{::}\KeywordTok{simulate_event_counts}\NormalTok{(}\DataTypeTok{process =}\NormalTok{ lambda,}
                                            \DataTypeTok{times =}\NormalTok{ times,}
                                            \DataTypeTok{nevents =}\NormalTok{ nevents,}
                                            \DataTypeTok{nsamples =}\NormalTok{ nsamples,}
                                            \DataTypeTok{binning_resolution =} \DecValTok{-10}\NormalTok{,}
                                            \DataTypeTok{bigmatrix =}\NormalTok{ datapath)}
\KeywordTok{save}\NormalTok{(simdata2,}
    \DataTypeTok{file =} \KeywordTok{paste}\NormalTok{(datapath,}
                \StringTok{"simdata2.RData"}\NormalTok{,}
                \DataTypeTok{sep =} \StringTok{""}\NormalTok{))}
\end{Highlighting}
\end{Shaded}

The \texttt{chronup} package also includes a convenient way to plot a
\texttt{rece} or count ensemble. To speed up the plotting process, we
will plot only a subset of the probable sequences. Since the ensemble
has been stored in a big memory matrix, we will first have to attach
that matrix. The path to the matrix is stored as an element in the
\texttt{simdata2} object (\texttt{simdata\$count\_ensemble}).

\begin{Shaded}
\begin{Highlighting}[]
\KeywordTok{load}\NormalTok{(}\KeywordTok{paste}\NormalTok{(datapath,}
        \StringTok{"simdata2.RData"}\NormalTok{,}
        \DataTypeTok{sep =} \StringTok{""}\NormalTok{))}

\NormalTok{rece <-}\StringTok{ }\NormalTok{bigmemory}\OperatorTok{::}\KeywordTok{attach.big.matrix}\NormalTok{(simdata2}\OperatorTok{$}\NormalTok{count_ensemble)}
\NormalTok{rece_sub <-}\StringTok{ }\NormalTok{rece[,}\DecValTok{1}\OperatorTok{:}\DecValTok{1000}\NormalTok{]}

\NormalTok{chronup}\OperatorTok{::}\KeywordTok{plot_count_ensemble}\NormalTok{(}\DataTypeTok{count_ensemble =}\NormalTok{ rece_sub,}
                            \DataTypeTok{times =}\NormalTok{ simdata2}\OperatorTok{$}\NormalTok{new_timestamps,}
                            \DataTypeTok{axis_x_res =} \DecValTok{10}\NormalTok{)}
\end{Highlighting}
\end{Shaded}

\includegraphics{replication_files/figure-latex/chronup-plot-1.pdf}

\begin{verbatim}
## NULL
\end{verbatim}

Chronological uncertainty is represented in the plot above by a heat
map. The colours indicate the relative probability of count-time
pairings based on the sample represented by the count ensemble. Warmer
colours indicate higher relative probabilities whereas cooler colours
indicate lower relative probabilities. So, the most probable true event
count sequence will be located somewhere in the hottest, brightest
region of the plot. By default, the plotting function presents the
passage of time moving left to right. Remember that this is effectively
a density estimate not a model, and it needs to be treated as such.

\section{Introducing Chronological Uncertainty in the
Covariate}\label{introducing-chronological-uncertainty-in-the-covariate}

In almost all practical Palaeo Science analyses both the count data and
any potential explanatory covariates will have chronological
uncertainties. The covariate (independent variable) used in the
simulation analyses described here is simply a linear function of
time---essentially just the passage of time itself. The values of the
covariate are, therefore, just a series of increasing integers, each
corresponding to one of the intervals into which the events fall. Still,
it would be useful to include some kind of covariate uncertainty since
it would be present in real Palaeo Science research.

With that in mind, we can create another ensemble matrix that reflects
covariate uncertainty. Each column in the matrix will refer to one
probable covariate sequence. These could be, for instance, probable
palaeoclimate reconstructions involving isotopes measured in a
layer-counted depositional sequence, like a speleothem or a lake core.
Each sequence, then, would be one probable sequence of isotope
measurements given the chronological and depth measurement uncertainties
in a relevant age-depth model. Since the covariate used in the present
analysis is a simple series of integers, some uncertainty can be
introduced by adding a random deviation to each value using R's built-in
probability distributions.

It should be noted that since the span of time occupied by the RECE
samples is longer than the span of the original process. So, we are
confronted with two possible solutions. One is to extend the simulated
covariate series forward and backward (start the sequence at 0 where
that corresponds to the first element of any given RECE sample). The
other is to crop both the dependent and independent variables to the
span of the true process defined by the \texttt{start} and \texttt{end}
parameters. For this simulation exercise the simplest solution is the
second one, so that is what we will do.

To begin, establish a bigmatrix object to hold the covariate samples.
The matrix will have the same number of rows as the RECE matrix and
twice the number of columns. We are using twice the number of columns in
order to include an intercept in the model, which will simply be a
column of \texttt{1}'s. As the MCMC proceeds, columns from the
independent variable matrix (\texttt{X}) will be selected two at a time
to account for the intercept.

\begin{Shaded}
\begin{Highlighting}[]
\NormalTok{X <-}\StringTok{ }\NormalTok{bigmemory}\OperatorTok{::}\KeywordTok{filebacked.big.matrix}\NormalTok{(}
                    \DataTypeTok{nrow =} \KeywordTok{dim}\NormalTok{(rece)[}\DecValTok{1}\NormalTok{],}
                    \DataTypeTok{ncol =} \KeywordTok{dim}\NormalTok{(rece)[}\DecValTok{2}\NormalTok{] }\OperatorTok{*}\StringTok{ }\DecValTok{2}\NormalTok{,}
                    \DataTypeTok{backingpath =}\NormalTok{ datapath,}
                    \DataTypeTok{backingfile =} \StringTok{"X_mat"}\NormalTok{,}
                    \DataTypeTok{descriptorfile =} \StringTok{"X_mat_desc"}\NormalTok{)}

\NormalTok{X_loc <-}\StringTok{ }\NormalTok{bigmemory}\OperatorTok{::}\KeywordTok{describe}\NormalTok{(X)}
\end{Highlighting}
\end{Shaded}

For each observation in the covariate sequence, we add a random
deviation and then store the result in the \texttt{X} matrix. We can
make use of parallelization and a progress bar to speed up this process
and monitor it. Note that in order to preserve the temporal scale and
obtain estimates of \texttt{beta} at the correct scale, we have to
multiply the time indeces by the resolution of the temporal bins, which
means multiplying in this case by 10. Likewise, it is important to note
that the uncertainty we wish to introduce into the covariate would need
to be scaled accordingly to ensure that a sensible level of uncertainty
is being included. For simplicity, we will use a standard deviation 2,
which corresponds to 20 years at the original temporal resolution of the
simulated data. In a real-world analysis, forgetting to account for the
temporal width of the bins can lead to scaled parameter estimates. The
estimates would be a multiple of the temporal resolution in the case of
a regression coefficient referring only to the passage of time.
Correcting the scale after obtaining parameter estimates would involve
dividing those estimates by the temporal resolution of the bins, but
here we can avoid the problem by scaling appropriately at the outset.

\begin{Shaded}
\begin{Highlighting}[]
\KeywordTok{library}\NormalTok{(pbapply)}
\KeywordTok{library}\NormalTok{(parallel)}

\NormalTok{cl <-}\StringTok{ }\KeywordTok{makeCluster}\NormalTok{(}\KeywordTok{detectCores}\NormalTok{() }\OperatorTok{-}\StringTok{ }\DecValTok{1}\NormalTok{)}

\NormalTok{x_error =}\StringTok{ }\DecValTok{2}

\KeywordTok{pbsapply}\NormalTok{(}\DataTypeTok{cl =}\NormalTok{ cl,}
        \DataTypeTok{X =} \DecValTok{1}\OperatorTok{:}\KeywordTok{dim}\NormalTok{(rece)[}\DecValTok{2}\NormalTok{],}
        \DataTypeTok{FUN =} \ControlFlowTok{function}\NormalTok{(j,}
\NormalTok{                    x_error,}
\NormalTok{                    n,}
\NormalTok{                    nX,}
\NormalTok{                    X_loc)\{}
                \KeywordTok{library}\NormalTok{(bigmemory)}
\NormalTok{                X <-}\StringTok{ }\KeywordTok{attach.big.matrix}\NormalTok{(X_loc)}
\NormalTok{                x_indeces <-}\StringTok{ }\DecValTok{1}\OperatorTok{:}\NormalTok{nX }\OperatorTok{+}\StringTok{ }\NormalTok{(nX }\OperatorTok{*}\StringTok{ }\NormalTok{(j }\OperatorTok{-}\StringTok{ }\DecValTok{1}\NormalTok{))}
\NormalTok{                X[, x_indeces[}\DecValTok{1}\NormalTok{]] <-}\StringTok{ }\DecValTok{1}
\NormalTok{                xx <-}\StringTok{ }\KeywordTok{rnorm}\NormalTok{(}\DataTypeTok{n =}\NormalTok{ n,}
                            \DataTypeTok{mean =} \DecValTok{0}\OperatorTok{:}\NormalTok{(n }\OperatorTok{-}\StringTok{ }\DecValTok{1}\NormalTok{),}
                            \DataTypeTok{sd =}\NormalTok{ x_error)}
\NormalTok{                X[,x_indeces[}\DecValTok{2}\NormalTok{]] <-}\StringTok{ }\NormalTok{xx }\OperatorTok{*}\StringTok{ }\DecValTok{10}
                \KeywordTok{return}\NormalTok{()\},}
        \DataTypeTok{x_error =}\NormalTok{ x_error,}
        \DataTypeTok{n =} \KeywordTok{dim}\NormalTok{(rece)[}\DecValTok{1}\NormalTok{],}
        \DataTypeTok{nX =} \DecValTok{2}\NormalTok{,}
        \DataTypeTok{X_loc =}\NormalTok{ X_loc)}

\KeywordTok{stopCluster}\NormalTok{(cl)}
\end{Highlighting}
\end{Shaded}

\section{Accounting for chronological uncertainty in
regression}\label{accounting-for-chronological-uncertainty-in-regression}

Recently, attempts have been made to handle chronological uncertainty in
Palaeo Science regression analyses. As described in the paper associated
with this replication document, these recent attempts have involved a
hierarchical arrangement of parameters in count-based regression models.
In this section, we run one of these analyses on the data simulated
above and then explore an alternate approach using the same data.

\subsection{Hierarchical arrangement}\label{hierarchical-arrangement}

For the hierarchical model, we use a Bayesian R package called
\href{https://r-nimble.org/}{Nimble}. Nimble is a system for creating
Bayesian models and estimating model parameters with a flexible,
sophisticated MCMC engine. The core of Nimble model specification is
based on an older language for writing Bayesian models called ``BUGS''.
The first step is to write the model using a Nimble R function and
Nimble's slightly adapted BUGS language. Then, the model definition
along with other important variables are used to create a Nimble model
object, which is subsequently passed to Nimble's MCMC engine (after MCMC
settings are established). Nimble allows for all of the relevant R code
and objects to be compiled to C++, which massively reduces computation
times.

The following code is used to create a hierarchical model in Nimble and
run an MCMC given the data simulated above,

\begin{Shaded}
\begin{Highlighting}[]
\KeywordTok{library}\NormalTok{(nimble)}

\KeywordTok{load}\NormalTok{(}\KeywordTok{paste}\NormalTok{(datapath,}
        \StringTok{"simdata2.RData"}\NormalTok{,}
        \DataTypeTok{sep =} \StringTok{""}\NormalTok{))}

\NormalTok{rece <-}\StringTok{ }\NormalTok{bigmemory}\OperatorTok{::}\KeywordTok{attach.big.matrix}\NormalTok{(simdata2}\OperatorTok{$}\NormalTok{count_ensemble)}
\NormalTok{X <-}\StringTok{ }\NormalTok{bigmemory}\OperatorTok{::}\KeywordTok{attach.big.matrix}\NormalTok{(}\KeywordTok{paste}\NormalTok{(datapath,}
                                \StringTok{"X_mat_desc"}\NormalTok{,}
                                \DataTypeTok{sep =} \StringTok{""}\NormalTok{))}

\NormalTok{nbCode <-}\StringTok{ }\KeywordTok{nimbleCode}\NormalTok{(\{}
\NormalTok{   ###top-level regression}
\NormalTok{   B0 }\OperatorTok{~}\StringTok{ }\KeywordTok{dnorm}\NormalTok{(}\DecValTok{0}\NormalTok{, }\DecValTok{100}\NormalTok{)}
\NormalTok{   B1 }\OperatorTok{~}\StringTok{ }\KeywordTok{dnorm}\NormalTok{(}\DecValTok{0}\NormalTok{, }\DecValTok{100}\NormalTok{)}
\NormalTok{   sigB0 }\OperatorTok{~}\StringTok{ }\KeywordTok{dunif}\NormalTok{(}\FloatTok{1e-10}\NormalTok{, }\DecValTok{100}\NormalTok{)}
\NormalTok{   sigB1 }\OperatorTok{~}\StringTok{ }\KeywordTok{dunif}\NormalTok{(}\FloatTok{1e-10}\NormalTok{, }\DecValTok{100}\NormalTok{)}
   \ControlFlowTok{for}\NormalTok{ (j }\ControlFlowTok{in} \DecValTok{1}\OperatorTok{:}\NormalTok{J) \{}
\NormalTok{      ###low-level regression}
\NormalTok{      b0[j] }\OperatorTok{~}\StringTok{ }\KeywordTok{dnorm}\NormalTok{(}\DataTypeTok{mean =}\NormalTok{ B0, }\DataTypeTok{sd =}\NormalTok{ sigB0)}
\NormalTok{      b1[j] }\OperatorTok{~}\StringTok{ }\KeywordTok{dnorm}\NormalTok{(}\DataTypeTok{mean =}\NormalTok{ B1, }\DataTypeTok{sd =}\NormalTok{ sigB1)}
      \ControlFlowTok{for}\NormalTok{ (n }\ControlFlowTok{in} \DecValTok{1}\OperatorTok{:}\NormalTok{N)\{}
\NormalTok{        p[n, j] }\OperatorTok{~}\StringTok{ }\KeywordTok{dunif}\NormalTok{(}\FloatTok{1e-10}\NormalTok{, }\DecValTok{1}\NormalTok{)}
\NormalTok{        r[n, j] <-}\StringTok{ }\KeywordTok{exp}\NormalTok{(b0[j] }\OperatorTok{+}\StringTok{ }\NormalTok{X[n, j] }\OperatorTok{*}\StringTok{ }\NormalTok{b1[j])}
\NormalTok{        Y[n, j] }\OperatorTok{~}\StringTok{ }\KeywordTok{dnegbin}\NormalTok{(}\DataTypeTok{size =}\NormalTok{ r[n, j], }\DataTypeTok{prob =}\NormalTok{ p[n, j])}
\NormalTok{      \}}
\NormalTok{   \}}
\NormalTok{\})}
\end{Highlighting}
\end{Shaded}

With the Nimble model defined, sample the data matrices, setup the MCMC,
and run it. First, we will create a vector (\texttt{span\_index}) for
subsetting the data to the interval of time covered by the simlations
process (i.e., 5000:4001 BP). We will also use Nimble's slice samplers
for the sub-regressions (probable regressions at the lower level of the
hierarchy) to improve the sampling of the model posteriors and then run
the MCMC for at least \texttt{niter\ =\ 1000000} iterations (thinned to
retain only every 10th sample). Some of these MCMC parameters can be set
up globally, while others have to be established again for each new MCMC
run because of the way Nimble compiles code to C++, as described
earlier. The global parameters are,

\begin{Shaded}
\begin{Highlighting}[]
\NormalTok{span_index <-}\StringTok{ }\KeywordTok{which}\NormalTok{(simdata2}\OperatorTok{$}\NormalTok{new_timestamps }\OperatorTok{<=}\StringTok{ }\NormalTok{start }\OperatorTok{&}
\StringTok{                    }\NormalTok{simdata2}\OperatorTok{$}\NormalTok{new_timestamps }\OperatorTok{>=}\StringTok{ }\NormalTok{end)}
\NormalTok{n <-}\StringTok{ }\KeywordTok{length}\NormalTok{(span_index)}

\CommentTok{#number of MCMC iterations}
\NormalTok{niter <-}\StringTok{ }\DecValTok{1000000}

\CommentTok{#thinning to save space}
\NormalTok{thinning <-}\StringTok{ }\DecValTok{10}

\CommentTok{#set seed for replicability}
\KeywordTok{set.seed}\NormalTok{(}\DecValTok{1}\NormalTok{)}
\end{Highlighting}
\end{Shaded}

To explore the impact of the size of the sample of probable sequences on
the posterior density estimates, we will run 4 separate MCMC
simulations: one simulation for each of \(J \in [5, 10, 50, 100]\). Only
the code for the first of these---\texttt{J\ =\ 5}---is printed in the
PDF/md document produced by this script, but the code for each is in the
script below.

Importantly, we need to select the J draws from the RECE and the
corresponding J draws from the matrix containing the covariate (X)
values with uncertainty added in. The J RECE columns will be randomly
sampled without replacement (they are already a random sample of
probable count sequences and we do not want to select a given sequence
more than once). For the `X' matrix, we need to select every other
column, though, because that matrix contains alternating columns of 1's
representing the model intercept and covariate values. The intercept is
included in the Nimble model specification internally, and so needs to
be excluded from the covariate matrix.

The code for the MCMC involving \texttt{J\ =\ 5} probable sequences (and
matching covariate sequences) is as follows,

\begin{Shaded}
\begin{Highlighting}[]
\NormalTok{span_index <-}\StringTok{ }\KeywordTok{which}\NormalTok{(simdata2}\OperatorTok{$}\NormalTok{new_timestamps }\OperatorTok{<=}\StringTok{ }\NormalTok{start }\OperatorTok{&}
\StringTok{                    }\NormalTok{simdata2}\OperatorTok{$}\NormalTok{new_timestamps }\OperatorTok{>=}\StringTok{ }\NormalTok{end)}
\NormalTok{n <-}\StringTok{ }\KeywordTok{length}\NormalTok{(span_index)}
\NormalTok{J =}\StringTok{ }\DecValTok{5}

\NormalTok{y_sample <-}\StringTok{ }\KeywordTok{sample}\NormalTok{(}\DecValTok{1}\OperatorTok{:}\KeywordTok{dim}\NormalTok{(rece)[}\DecValTok{2}\NormalTok{],}
                \DataTypeTok{size =}\NormalTok{ J,}
                \DataTypeTok{replace =}\NormalTok{ F)}
\NormalTok{x_sample <-}\StringTok{ }\KeywordTok{sample}\NormalTok{(}\KeywordTok{seq}\NormalTok{(}\DecValTok{2}\NormalTok{, }\KeywordTok{dim}\NormalTok{(rece)[}\DecValTok{2}\NormalTok{], }\DecValTok{2}\NormalTok{),}
                \DataTypeTok{size =}\NormalTok{ J,}
                \DataTypeTok{replace =}\NormalTok{ F)}

\NormalTok{nbData <-}\StringTok{ }\KeywordTok{list}\NormalTok{(}\DataTypeTok{Y =}\NormalTok{ rece[span_index, y_sample],}
                \DataTypeTok{X =}\NormalTok{ X[span_index, x_sample])}

\NormalTok{nbConsts <-}\StringTok{ }\KeywordTok{list}\NormalTok{(}\DataTypeTok{N =}\NormalTok{ n,}
                \DataTypeTok{J =}\NormalTok{ J)}

\NormalTok{nbInits <-}\StringTok{ }\KeywordTok{list}\NormalTok{(}\DataTypeTok{B0 =} \DecValTok{0}\NormalTok{,}
                \DataTypeTok{B1 =} \DecValTok{0}\NormalTok{,}
                \DataTypeTok{b0 =} \KeywordTok{rep}\NormalTok{(}\DecValTok{0}\NormalTok{, J),}
                \DataTypeTok{b1 =} \KeywordTok{rep}\NormalTok{(}\DecValTok{0}\NormalTok{, J),}
                \DataTypeTok{sigB0 =} \FloatTok{0.0001}\NormalTok{,}
                \DataTypeTok{sigB1 =} \FloatTok{0.0001}\NormalTok{)}

\NormalTok{nbModel <-}\StringTok{ }\KeywordTok{nimbleModel}\NormalTok{(}\DataTypeTok{code =}\NormalTok{ nbCode,}
                        \DataTypeTok{data =}\NormalTok{ nbData,}
                        \DataTypeTok{inits =}\NormalTok{ nbInits,}
                        \DataTypeTok{constants =}\NormalTok{ nbConsts)}

\CommentTok{#compile nimble model to C++ code—much faster runtime}
\NormalTok{C_nbModel <-}\StringTok{ }\KeywordTok{compileNimble}\NormalTok{(nbModel, }\DataTypeTok{showCompilerOutput =} \OtherTok{FALSE}\NormalTok{)}

\CommentTok{#configure the MCMC}
\NormalTok{nbModel_conf <-}\StringTok{ }\KeywordTok{configureMCMC}\NormalTok{(nbModel)}

\NormalTok{nbModel_conf}\OperatorTok{$}\NormalTok{monitors <-}\StringTok{ }\KeywordTok{c}\NormalTok{(}\StringTok{"B0"}\NormalTok{, }\StringTok{"B1"}\NormalTok{, }\StringTok{"sigB0"}\NormalTok{, }\StringTok{"sigB1"}\NormalTok{)}
\NormalTok{nbModel_conf}\OperatorTok{$}\KeywordTok{addMonitors2}\NormalTok{(}\KeywordTok{c}\NormalTok{(}\StringTok{"b0"}\NormalTok{, }\StringTok{"b1"}\NormalTok{))}

\CommentTok{#samplers}
\NormalTok{nbModel_conf}\OperatorTok{$}\KeywordTok{removeSamplers}\NormalTok{(}\KeywordTok{c}\NormalTok{(}\StringTok{"B0"}\NormalTok{, }\StringTok{"B1"}\NormalTok{, }\StringTok{"sigB0"}\NormalTok{, }\StringTok{"sigB1"}\NormalTok{, }\StringTok{"b0"}\NormalTok{, }\StringTok{"b1"}\NormalTok{))}
\NormalTok{nbModel_conf}\OperatorTok{$}\KeywordTok{addSampler}\NormalTok{(}\DataTypeTok{target =} \KeywordTok{c}\NormalTok{(}\StringTok{"B0"}\NormalTok{, }\StringTok{"B1"}\NormalTok{, }\StringTok{"sigB0"}\NormalTok{, }\StringTok{"sigB1"}\NormalTok{),}
                        \DataTypeTok{type =} \StringTok{"AF_slice"}\NormalTok{)}
\ControlFlowTok{for}\NormalTok{(j }\ControlFlowTok{in} \DecValTok{1}\OperatorTok{:}\NormalTok{J)\{}
\NormalTok{   nbModel_conf}\OperatorTok{$}\KeywordTok{addSampler}\NormalTok{(}\DataTypeTok{target =} \KeywordTok{c}\NormalTok{(}\KeywordTok{paste}\NormalTok{(}\StringTok{"b0["}\NormalTok{, j, }\StringTok{"]"}\NormalTok{, }\DataTypeTok{sep =} \StringTok{""}\NormalTok{),}
                                    \KeywordTok{paste}\NormalTok{(}\StringTok{"b1["}\NormalTok{, j, }\StringTok{"]"}\NormalTok{, }\DataTypeTok{sep =} \StringTok{""}\NormalTok{)),}
                        \DataTypeTok{type =} \StringTok{"AF_slice"}\NormalTok{)}
\NormalTok{\}}

\CommentTok{#thinning to conserve memory when the samples are saved below}
\NormalTok{nbModel_conf}\OperatorTok{$}\KeywordTok{setThin}\NormalTok{(thinning)}
\NormalTok{nbModel_conf}\OperatorTok{$}\KeywordTok{setThin2}\NormalTok{(thinning)}

\CommentTok{#build MCMC}
\NormalTok{nbModelMCMC <-}\StringTok{ }\KeywordTok{buildMCMC}\NormalTok{(nbModel_conf)}

\CommentTok{#compile MCMC to C++—much faster}
\NormalTok{C_nbModelMCMC <-}\StringTok{ }\KeywordTok{compileNimble}\NormalTok{(nbModelMCMC, }\DataTypeTok{project =}\NormalTok{ nbModel)}

\CommentTok{#save the MCMC chain (monitored variables) as a matrix}
\NormalTok{nimble_samples_}\DecValTok{5}\NormalTok{ <-}\StringTok{ }\KeywordTok{runMCMC}\NormalTok{(C_nbModelMCMC, }\DataTypeTok{niter =}\NormalTok{ niter)}

\CommentTok{#save samples}
\KeywordTok{save}\NormalTok{(nimble_samples_}\DecValTok{5}\NormalTok{,}
        \DataTypeTok{file =} \KeywordTok{paste}\NormalTok{(datapath,}
                    \StringTok{"mcmc_samples_nimble_5.RData"}\NormalTok{,}
                    \DataTypeTok{sep =} \StringTok{""}\NormalTok{))}
\end{Highlighting}
\end{Shaded}

\begin{Shaded}
\begin{Highlighting}[]
\NormalTok{span_index <-}\StringTok{ }\KeywordTok{which}\NormalTok{(simdata2}\OperatorTok{$}\NormalTok{new_timestamps }\OperatorTok{<=}\StringTok{ }\NormalTok{start }\OperatorTok{&}
\StringTok{                    }\NormalTok{simdata2}\OperatorTok{$}\NormalTok{new_timestamps }\OperatorTok{>=}\StringTok{ }\NormalTok{end)}
\NormalTok{n <-}\StringTok{ }\KeywordTok{length}\NormalTok{(span_index)}
\NormalTok{J =}\StringTok{ }\DecValTok{50}

\NormalTok{y_sample <-}\StringTok{ }\KeywordTok{sample}\NormalTok{(}\DecValTok{1}\OperatorTok{:}\KeywordTok{dim}\NormalTok{(rece)[}\DecValTok{2}\NormalTok{],}
                \DataTypeTok{size =}\NormalTok{ J,}
                \DataTypeTok{replace =}\NormalTok{ F)}
\NormalTok{x_sample <-}\StringTok{ }\KeywordTok{sample}\NormalTok{(}\KeywordTok{seq}\NormalTok{(}\DecValTok{2}\NormalTok{, }\KeywordTok{dim}\NormalTok{(rece)[}\DecValTok{2}\NormalTok{], }\DecValTok{2}\NormalTok{),}
                \DataTypeTok{size =}\NormalTok{ J,}
                \DataTypeTok{replace =}\NormalTok{ F)}

\NormalTok{nbData <-}\StringTok{ }\KeywordTok{list}\NormalTok{(}\DataTypeTok{Y =}\NormalTok{ rece[span_index, y_sample],}
                \DataTypeTok{X =}\NormalTok{ X[span_index, x_sample])}

\NormalTok{nbConsts <-}\StringTok{ }\KeywordTok{list}\NormalTok{(}\DataTypeTok{N =}\NormalTok{ n,}
                \DataTypeTok{J =}\NormalTok{ J)}

\NormalTok{nbInits <-}\StringTok{ }\KeywordTok{list}\NormalTok{(}\DataTypeTok{B0 =} \DecValTok{0}\NormalTok{,}
                \DataTypeTok{B1 =} \DecValTok{0}\NormalTok{,}
                \DataTypeTok{b0 =} \KeywordTok{rep}\NormalTok{(}\DecValTok{0}\NormalTok{, J),}
                \DataTypeTok{b1 =} \KeywordTok{rep}\NormalTok{(}\DecValTok{0}\NormalTok{, J),}
                \DataTypeTok{sigB0 =} \FloatTok{0.0001}\NormalTok{,}
                \DataTypeTok{sigB1 =} \FloatTok{0.0001}\NormalTok{)}

\NormalTok{nbModel <-}\StringTok{ }\KeywordTok{nimbleModel}\NormalTok{(}\DataTypeTok{code =}\NormalTok{ nbCode,}
                        \DataTypeTok{data =}\NormalTok{ nbData,}
                        \DataTypeTok{inits =}\NormalTok{ nbInits,}
                        \DataTypeTok{constants =}\NormalTok{ nbConsts)}

\CommentTok{#compile nimble model to C++ code—much faster runtime}
\NormalTok{C_nbModel <-}\StringTok{ }\KeywordTok{compileNimble}\NormalTok{(nbModel, }\DataTypeTok{showCompilerOutput =} \OtherTok{FALSE}\NormalTok{)}

\CommentTok{#configure the MCMC}
\NormalTok{nbModel_conf <-}\StringTok{ }\KeywordTok{configureMCMC}\NormalTok{(nbModel)}

\NormalTok{nbModel_conf}\OperatorTok{$}\NormalTok{monitors <-}\StringTok{ }\KeywordTok{c}\NormalTok{(}\StringTok{"B0"}\NormalTok{, }\StringTok{"B1"}\NormalTok{, }\StringTok{"sigB0"}\NormalTok{, }\StringTok{"sigB1"}\NormalTok{)}
\NormalTok{nbModel_conf}\OperatorTok{$}\KeywordTok{addMonitors2}\NormalTok{(}\KeywordTok{c}\NormalTok{(}\StringTok{"b0"}\NormalTok{, }\StringTok{"b1"}\NormalTok{))}

\CommentTok{#samplers}
\NormalTok{nbModel_conf}\OperatorTok{$}\KeywordTok{removeSamplers}\NormalTok{(}\KeywordTok{c}\NormalTok{(}\StringTok{"B0"}\NormalTok{, }\StringTok{"B1"}\NormalTok{, }\StringTok{"sigB0"}\NormalTok{, }\StringTok{"sigB1"}\NormalTok{, }\StringTok{"b0"}\NormalTok{, }\StringTok{"b1"}\NormalTok{))}
\NormalTok{nbModel_conf}\OperatorTok{$}\KeywordTok{addSampler}\NormalTok{(}\DataTypeTok{target =} \KeywordTok{c}\NormalTok{(}\StringTok{"B0"}\NormalTok{, }\StringTok{"B1"}\NormalTok{, }\StringTok{"sigB0"}\NormalTok{, }\StringTok{"sigB1"}\NormalTok{),}
                        \DataTypeTok{type =} \StringTok{"AF_slice"}\NormalTok{)}
\ControlFlowTok{for}\NormalTok{(j }\ControlFlowTok{in} \DecValTok{1}\OperatorTok{:}\NormalTok{J)\{}
\NormalTok{   nbModel_conf}\OperatorTok{$}\KeywordTok{addSampler}\NormalTok{(}\DataTypeTok{target =} \KeywordTok{c}\NormalTok{(}\KeywordTok{paste}\NormalTok{(}\StringTok{"b0["}\NormalTok{, j, }\StringTok{"]"}\NormalTok{, }\DataTypeTok{sep =} \StringTok{""}\NormalTok{),}
                                    \KeywordTok{paste}\NormalTok{(}\StringTok{"b1["}\NormalTok{, j, }\StringTok{"]"}\NormalTok{, }\DataTypeTok{sep =} \StringTok{""}\NormalTok{)),}
                        \DataTypeTok{type =} \StringTok{"AF_slice"}\NormalTok{)}
\NormalTok{\}}

\CommentTok{#thinning to conserve memory when the samples are saved below}
\NormalTok{nbModel_conf}\OperatorTok{$}\KeywordTok{setThin}\NormalTok{(thinning)}
\NormalTok{nbModel_conf}\OperatorTok{$}\KeywordTok{setThin2}\NormalTok{(thinning)}

\CommentTok{#build MCMC}
\NormalTok{nbModelMCMC <-}\StringTok{ }\KeywordTok{buildMCMC}\NormalTok{(nbModel_conf)}

\CommentTok{#compile MCMC to C++—much faster}
\NormalTok{C_nbModelMCMC <-}\StringTok{ }\KeywordTok{compileNimble}\NormalTok{(nbModelMCMC, }\DataTypeTok{project =}\NormalTok{ nbModel)}

\CommentTok{#save the MCMC chain (monitored variables) as a matrix}
\NormalTok{nimble_samples_}\DecValTok{50}\NormalTok{ <-}\StringTok{ }\KeywordTok{runMCMC}\NormalTok{(C_nbModelMCMC, }\DataTypeTok{niter =}\NormalTok{ niter)}

\CommentTok{#save samples}
\KeywordTok{save}\NormalTok{(nimble_samples_}\DecValTok{50}\NormalTok{,}
        \DataTypeTok{file =} \KeywordTok{paste}\NormalTok{(datapath,}
                    \StringTok{"mcmc_samples_nimble_50.RData"}\NormalTok{,}
                    \DataTypeTok{sep =} \StringTok{""}\NormalTok{))}
\end{Highlighting}
\end{Shaded}

\subsection{\texorpdfstring{Alternative with
\texttt{chronup}}{Alternative with chronup}}\label{alternative-with-chronup}

Next, we can use a regression function from the \texttt{chronup} package
to perform another count-based regression on these simulated data while
propagating the uncertainty in the data up to the posterior estimates of
the model's parameters. In this case, though, the parameters are not
hierarchically arranged as they were for the Nimble analysis.

\begin{Shaded}
\begin{Highlighting}[]
\NormalTok{span_index <-}\StringTok{ }\KeywordTok{which}\NormalTok{(simdata2}\OperatorTok{$}\NormalTok{new_timestamps }\OperatorTok{<=}\StringTok{ }\NormalTok{start }\OperatorTok{&}
\StringTok{                    }\NormalTok{simdata2}\OperatorTok{$}\NormalTok{new_timestamps }\OperatorTok{>=}\StringTok{ }\NormalTok{end)}

\NormalTok{startvals <-}\StringTok{ }\KeywordTok{c}\NormalTok{(}\DecValTok{0}\NormalTok{, }\DecValTok{0}\NormalTok{, }\KeywordTok{rep}\NormalTok{(}\FloatTok{0.1}\NormalTok{, }\DecValTok{100}\NormalTok{))}
\NormalTok{startscales <-}\StringTok{ }\KeywordTok{c}\NormalTok{(}\FloatTok{0.1}\NormalTok{, }\FloatTok{0.0002}\NormalTok{, }\KeywordTok{rep}\NormalTok{(}\FloatTok{0.5}\NormalTok{, }\DecValTok{100}\NormalTok{))}

\NormalTok{adapt_limit <-}\StringTok{ }\KeywordTok{floor}\NormalTok{(nsamples }\OperatorTok{*}\StringTok{ }\FloatTok{0.25}\NormalTok{)}

\NormalTok{adapt_sample_y <-}\StringTok{ }\DecValTok{1}\OperatorTok{:}\NormalTok{adapt_limit}
\NormalTok{adapt_sample_x <-}\StringTok{ }\DecValTok{1}\OperatorTok{:}\NormalTok{(adapt_limit }\OperatorTok{*}\StringTok{ }\DecValTok{2}\NormalTok{)}

\NormalTok{mcmc_samples_adapt <-}\StringTok{ }\NormalTok{chronup}\OperatorTok{::}\KeywordTok{regress}\NormalTok{(}\DataTypeTok{Y =}\NormalTok{ rece[span_index, adapt_sample_y],}
                                        \DataTypeTok{X =}\NormalTok{ X[span_index, adapt_sample_x],}
                                        \DataTypeTok{model =} \StringTok{"nb"}\NormalTok{,}
                                        \DataTypeTok{startvals =}\NormalTok{ startvals,}
                                        \DataTypeTok{scales =}\NormalTok{ startscales,}
                                        \DataTypeTok{adapt =}\NormalTok{ T)}

\NormalTok{burnin <-}\StringTok{ }\KeywordTok{floor}\NormalTok{(}\KeywordTok{dim}\NormalTok{(mcmc_samples_adapt}\OperatorTok{$}\NormalTok{samples)[}\DecValTok{1}\NormalTok{] }\OperatorTok{*}\StringTok{ }\FloatTok{0.2}\NormalTok{)}
\NormalTok{indeces <-}\StringTok{ }\KeywordTok{seq}\NormalTok{(burnin, }\KeywordTok{dim}\NormalTok{(mcmc_samples_adapt}\OperatorTok{$}\NormalTok{samples)[}\DecValTok{1}\NormalTok{], }\DecValTok{1}\NormalTok{)}
\NormalTok{adapt_startvals <-}\StringTok{ }\KeywordTok{colMeans}\NormalTok{(mcmc_samples_adapt}\OperatorTok{$}\NormalTok{samples[indeces, ])}

\NormalTok{burnin <-}\StringTok{ }\KeywordTok{floor}\NormalTok{(}\KeywordTok{dim}\NormalTok{(mcmc_samples_adapt}\OperatorTok{$}\NormalTok{scales)[}\DecValTok{1}\NormalTok{] }\OperatorTok{*}\StringTok{ }\FloatTok{0.2}\NormalTok{)}
\NormalTok{indeces <-}\StringTok{ }\KeywordTok{seq}\NormalTok{(burnin, }\KeywordTok{dim}\NormalTok{(mcmc_samples_adapt}\OperatorTok{$}\NormalTok{scales)[}\DecValTok{1}\NormalTok{], }\DecValTok{1}\NormalTok{)}
\NormalTok{adapt_startscales <-}\StringTok{ }\KeywordTok{colMeans}\NormalTok{(mcmc_samples_adapt}\OperatorTok{$}\NormalTok{scales[indeces, ])}

\NormalTok{chronup_samples <-}\StringTok{ }\NormalTok{chronup}\OperatorTok{::}\KeywordTok{regress}\NormalTok{(}\DataTypeTok{Y =}\NormalTok{ rece[span_index, ],}
                                \DataTypeTok{X =}\NormalTok{ X[span_index, ],}
                                \DataTypeTok{model =} \StringTok{"nb"}\NormalTok{,}
                                \DataTypeTok{startvals =}\NormalTok{ adapt_startvals,}
                                \DataTypeTok{scales =}\NormalTok{ adapt_startscales,}
                                \DataTypeTok{adapt =}\NormalTok{ F)}
\KeywordTok{save}\NormalTok{(chronup_samples,}
        \DataTypeTok{file =} \KeywordTok{paste}\NormalTok{(datapath,}
                    \StringTok{"mcmc_samples_chronup.RData"}\NormalTok{,}
                    \DataTypeTok{sep =} \StringTok{""}\NormalTok{))}
\end{Highlighting}
\end{Shaded}

\subsection{Comparing results}\label{comparing-results}

Finally, we can compare the results of the two approaches. The key
parameter of interest is \texttt{beta}. It is the main regression
coefficient in the models and represents the rate of the conditional
counting process. In a practical case, it would indicate the growth rate
(or, conversely, decay rate) of whatever process produced concentrations
of radiocarbon samples in the sediments excavated during a Palaeo
Science research project. If covariates other than time are involved,
the regression coefficients would reflect the impact of those covariates
on that growth/decay rate.

To see clearly how the two different approaches perform, we will compare
density estimates based on the posterior samples from the various MCMC
simulations. Using the following code, a side-by-side comparison can be
plotted with the help of \texttt{ggplot2}. First, though, we will
combine the samples into a single long dataframe. We will also discard
the first 10\% of samples from each as burn-in and scale the density
estimates to a maximum of one to make visual comparisons easier.
Initially, plot the densities without any x-axis constraints, but then
produce the same plot again with the x-axis limits set to focus on the
area around the target parameter estimate to facilitate comparison.

\begin{Shaded}
\begin{Highlighting}[]
\KeywordTok{load}\NormalTok{(}\KeywordTok{paste}\NormalTok{(datapath,}
            \StringTok{"mcmc_samples_nimble_5.RData"}\NormalTok{,}
            \DataTypeTok{sep =} \StringTok{""}\NormalTok{))}
\KeywordTok{load}\NormalTok{(}\KeywordTok{paste}\NormalTok{(datapath,}
            \StringTok{"mcmc_samples_nimble_10.RData"}\NormalTok{,}
            \DataTypeTok{sep =} \StringTok{""}\NormalTok{))}
\KeywordTok{load}\NormalTok{(}\KeywordTok{paste}\NormalTok{(datapath,}
            \StringTok{"mcmc_samples_nimble_50.RData"}\NormalTok{,}
            \DataTypeTok{sep =} \StringTok{""}\NormalTok{))}
\KeywordTok{load}\NormalTok{(}\KeywordTok{paste}\NormalTok{(datapath,}
            \StringTok{"mcmc_samples_nimble_100.RData"}\NormalTok{,}
            \DataTypeTok{sep =} \StringTok{""}\NormalTok{))}
\KeywordTok{load}\NormalTok{(}\KeywordTok{paste}\NormalTok{(datapath,}
            \StringTok{"mcmc_samples_chronup.RData"}\NormalTok{,}
            \DataTypeTok{sep =} \StringTok{""}\NormalTok{))}

\NormalTok{n_hier <-}\StringTok{ }\KeywordTok{dim}\NormalTok{(nimble_samples_}\DecValTok{5}\OperatorTok{$}\NormalTok{samples)[}\DecValTok{1}\NormalTok{]}
\NormalTok{n_alt <-}\StringTok{ }\KeywordTok{dim}\NormalTok{(chronup_samples)[}\DecValTok{1}\NormalTok{]}

\NormalTok{burnin_hier <-}\StringTok{ }\KeywordTok{floor}\NormalTok{(n_hier }\OperatorTok{*}\StringTok{ }\FloatTok{0.1}\NormalTok{)}
\NormalTok{burnin_alt <-}\StringTok{ }\KeywordTok{floor}\NormalTok{(n_alt }\OperatorTok{*}\StringTok{ }\FloatTok{0.1}\NormalTok{)}

\NormalTok{iteration_hier <-}\StringTok{ }\NormalTok{(burnin_hier }\OperatorTok{+}\StringTok{ }\DecValTok{1}\NormalTok{)}\OperatorTok{:}\NormalTok{n_hier}
\NormalTok{iteration_hier_all <-}\StringTok{ }\KeywordTok{rep}\NormalTok{(iteration_hier, }\DecValTok{4}\NormalTok{)}
\NormalTok{iteration_alt <-}\StringTok{ }\NormalTok{(burnin_alt }\OperatorTok{+}\StringTok{ }\DecValTok{1}\NormalTok{)}\OperatorTok{:}\NormalTok{n_alt}

\NormalTok{n_retained_hier <-}\StringTok{ }\KeywordTok{length}\NormalTok{(iteration_hier)}
\NormalTok{n_retained_alt <-}\StringTok{ }\KeywordTok{length}\NormalTok{(iteration_alt)}

\NormalTok{plot_samples_hier_}\DecValTok{5}\NormalTok{ <-}\StringTok{ }\NormalTok{nimble_samples_}\DecValTok{5}\OperatorTok{$}\NormalTok{samples[iteration_hier, }\DecValTok{2}\NormalTok{]}
\NormalTok{plot_samples_hier_}\DecValTok{10}\NormalTok{ <-}\StringTok{ }\NormalTok{nimble_samples_}\DecValTok{10}\OperatorTok{$}\NormalTok{samples[iteration_hier, }\DecValTok{2}\NormalTok{]}
\NormalTok{plot_samples_hier_}\DecValTok{50}\NormalTok{ <-}\StringTok{ }\NormalTok{nimble_samples_}\DecValTok{50}\OperatorTok{$}\NormalTok{samples[iteration_hier, }\DecValTok{2}\NormalTok{]}
\NormalTok{plot_samples_hier_}\DecValTok{100}\NormalTok{ <-}\StringTok{ }\NormalTok{nimble_samples_}\DecValTok{100}\OperatorTok{$}\NormalTok{samples[iteration_hier, }\DecValTok{2}\NormalTok{]}
\NormalTok{plot_samples_hier <-}\StringTok{ }\KeywordTok{c}\NormalTok{(plot_samples_hier_}\DecValTok{5}\NormalTok{,}
\NormalTok{                    plot_samples_hier_}\DecValTok{10}\NormalTok{,}
\NormalTok{                    plot_samples_hier_}\DecValTok{50}\NormalTok{,}
\NormalTok{                    plot_samples_hier_}\DecValTok{100}\NormalTok{)}
\NormalTok{plot_samples_alt <-}\StringTok{ }\NormalTok{chronup_samples[iteration_alt, }\DecValTok{2}\NormalTok{]}

\NormalTok{model_name_hier <-}\StringTok{ }\KeywordTok{rep}\NormalTok{(}\StringTok{"hier"}\NormalTok{, n_retained_hier }\OperatorTok{*}\StringTok{ }\DecValTok{4}\NormalTok{)}
\NormalTok{model_name_alt <-}\StringTok{ }\KeywordTok{rep}\NormalTok{(}\StringTok{"alt"}\NormalTok{, n_retained_alt)}

\NormalTok{J_hier <-}\StringTok{ }\KeywordTok{rep}\NormalTok{(}\KeywordTok{c}\NormalTok{(}\StringTok{"5"}\NormalTok{, }\StringTok{"10"}\NormalTok{, }\StringTok{"50"}\NormalTok{, }\StringTok{"100"}\NormalTok{), }\DataTypeTok{each =}\NormalTok{ n_retained_hier)}
\NormalTok{J_alt <-}\StringTok{ }\KeywordTok{rep}\NormalTok{(}\StringTok{"200000"}\NormalTok{, n_retained_alt)}

\NormalTok{J_both <-}\StringTok{ }\KeywordTok{factor}\NormalTok{(}\KeywordTok{c}\NormalTok{(J_hier, J_alt), }\DataTypeTok{levels =} \KeywordTok{c}\NormalTok{(}\StringTok{"5"}\NormalTok{, }\StringTok{"10"}\NormalTok{, }\StringTok{"50"}\NormalTok{, }\StringTok{"100"}\NormalTok{, }\StringTok{"200000"}\NormalTok{))}

\NormalTok{mcmc_samples_both <-}\StringTok{ }\KeywordTok{data.frame}\NormalTok{(}\DataTypeTok{Iteration =} \KeywordTok{c}\NormalTok{(iteration_hier_all,}
\NormalTok{                                            iteration_alt),}
                            \DataTypeTok{Model =} \KeywordTok{c}\NormalTok{(model_name_hier,}
\NormalTok{                                    model_name_alt),}
                            \DataTypeTok{J =}\NormalTok{ J_both,}
                            \DataTypeTok{Beta =} \KeywordTok{c}\NormalTok{(plot_samples_hier,}
\NormalTok{                                    plot_samples_alt))}

\KeywordTok{ggplot}\NormalTok{(}\DataTypeTok{data =}\NormalTok{ mcmc_samples_both) }\OperatorTok{+}
\StringTok{    }\KeywordTok{geom_vline}\NormalTok{(}\DataTypeTok{xintercept =}\NormalTok{ beta,}
                \DataTypeTok{linetype =} \DecValTok{2}\NormalTok{,}
                \DataTypeTok{alpha =} \FloatTok{0.8}\NormalTok{) }\OperatorTok{+}
\StringTok{    }\KeywordTok{geom_density}\NormalTok{(}\DataTypeTok{mapping =} \KeywordTok{aes}\NormalTok{(}\DataTypeTok{y =}\NormalTok{ ..scaled.., }\DataTypeTok{x =}\NormalTok{ Beta),}
                \DataTypeTok{fill =} \StringTok{"steelblue"}\NormalTok{,}
                \DataTypeTok{alpha =} \FloatTok{0.5}\NormalTok{) }\OperatorTok{+}
\StringTok{    }\KeywordTok{facet_wrap}\NormalTok{(}\OperatorTok{~}\StringTok{ }\NormalTok{J, }\DataTypeTok{ncol =} \DecValTok{2}\NormalTok{, }\DataTypeTok{scales =} \StringTok{"free"}\NormalTok{) }\OperatorTok{+}
\StringTok{    }\KeywordTok{theme_minimal}\NormalTok{()}
\end{Highlighting}
\end{Shaded}

\includegraphics{replication_files/figure-latex/plot-mcmc-densities-1.pdf}

To make comparisons easier, we can re-produce the same plot, but this
time limit the x-axis to a region around the target value for \(\beta\)
and align the plot axes:

\begin{Shaded}
\begin{Highlighting}[]
\KeywordTok{ggplot}\NormalTok{(}\DataTypeTok{data =}\NormalTok{ mcmc_samples_both) }\OperatorTok{+}
\StringTok{    }\KeywordTok{geom_vline}\NormalTok{(}\DataTypeTok{xintercept =}\NormalTok{ beta,}
                \DataTypeTok{linetype =} \DecValTok{2}\NormalTok{,}
                \DataTypeTok{alpha =} \FloatTok{0.8}\NormalTok{) }\OperatorTok{+}
\StringTok{    }\KeywordTok{geom_density}\NormalTok{(}\DataTypeTok{mapping =} \KeywordTok{aes}\NormalTok{(}\DataTypeTok{y =}\NormalTok{ ..scaled.., }\DataTypeTok{x =}\NormalTok{ Beta),}
                \DataTypeTok{fill =} \StringTok{"steelblue"}\NormalTok{,}
                \DataTypeTok{alpha =} \FloatTok{0.5}\NormalTok{) }\OperatorTok{+}
\StringTok{    }\KeywordTok{facet_wrap}\NormalTok{(}\OperatorTok{~}\StringTok{ }\NormalTok{J, }\DataTypeTok{ncol =} \DecValTok{2}\NormalTok{, }\DataTypeTok{scales =} \StringTok{"free"}\NormalTok{) }\OperatorTok{+}
\StringTok{    }\KeywordTok{scale_x_continuous}\NormalTok{(}\DataTypeTok{limits =} \KeywordTok{c}\NormalTok{(}\FloatTok{0.001}\NormalTok{, }\FloatTok{0.007}\NormalTok{)) }\OperatorTok{+}
\StringTok{    }\KeywordTok{theme_minimal}\NormalTok{()}
\end{Highlighting}
\end{Shaded}

\begin{verbatim}
## Warning: Removed 31 rows containing non-finite values (stat_density).
\end{verbatim}

\includegraphics{replication_files/figure-latex/plot-mcmc-densities2-1.pdf}

Lastly, we should also inspect the mcmc trace plots for each
model/estimate, and then check the Geweke statistics for the main
parameters,

\begin{Shaded}
\begin{Highlighting}[]
\KeywordTok{ggplot}\NormalTok{(}\DataTypeTok{data =}\NormalTok{ mcmc_samples_both) }\OperatorTok{+}
\StringTok{    }\KeywordTok{geom_hline}\NormalTok{(}\DataTypeTok{yintercept =}\NormalTok{ beta,}
                \DataTypeTok{linetype =} \DecValTok{2}\NormalTok{,}
                \DataTypeTok{alpha =} \FloatTok{0.8}\NormalTok{) }\OperatorTok{+}
\StringTok{    }\KeywordTok{geom_path}\NormalTok{(}\DataTypeTok{mapping =} \KeywordTok{aes}\NormalTok{(}\DataTypeTok{y =}\NormalTok{ Beta, }\DataTypeTok{x =}\NormalTok{ Iteration),}
                \DataTypeTok{alpha =} \FloatTok{0.85}\NormalTok{) }\OperatorTok{+}
\StringTok{    }\KeywordTok{facet_wrap}\NormalTok{(}\OperatorTok{~}\StringTok{ }\NormalTok{J, }\DataTypeTok{ncol =} \DecValTok{2}\NormalTok{, }\DataTypeTok{scales =} \StringTok{"free"}\NormalTok{) }\OperatorTok{+}
\StringTok{    }\KeywordTok{theme_minimal}\NormalTok{()}
\end{Highlighting}
\end{Shaded}

\includegraphics{replication_files/figure-latex/plot-mcmc-trace-1.pdf}

\begin{Shaded}
\begin{Highlighting}[]
\KeywordTok{geweke.diag}\NormalTok{(}\KeywordTok{cbind}\NormalTok{(plot_samples_hier_}\DecValTok{5}\NormalTok{,}
\NormalTok{                plot_samples_hier_}\DecValTok{10}\NormalTok{,}
\NormalTok{                plot_samples_hier_}\DecValTok{50}\NormalTok{,}
\NormalTok{                plot_samples_hier_}\DecValTok{100}\NormalTok{,}
\NormalTok{                plot_samples_alt))}
\end{Highlighting}
\end{Shaded}

\begin{verbatim}
## Warning in cbind(plot_samples_hier_5, plot_samples_hier_10,
## plot_samples_hier_50, : number of rows of result is not a multiple of vector
## length (arg 1)
\end{verbatim}

\begin{verbatim}
## 
## Fraction in 1st window = 0.1
## Fraction in 2nd window = 0.5 
## 
##   plot_samples_hier_5  plot_samples_hier_10  plot_samples_hier_50 
##                0.5707               -0.4789                0.5714 
## plot_samples_hier_100      plot_samples_alt 
##                0.8431                0.1215
\end{verbatim}

As the plots show, both methods yielded estimates for \texttt{beta} that
either include the true value (0.004, indicated by the vertical dashed
line) within their 99\% and/or 95\% credible intervals or are at least
quite close. But, the hierarchical model (\texttt{heir} in the plot
legend) produced posterior density estimates with a much lower variance
than the alternate method and can occasionally exclude the target value
when large samples of probable sequences are involved (i.e., \texttt{J}
is large).

\subsection{Sample Size}\label{sample-size}

A feature of hierarchical models is that they allow for borrowing of
information across samples. This leads to a phenomenon sometimes called
``shrinkage'' whereby the variance in the posterior density for model
parameters tends to be smaller than the variance of the same parameters
estimated separately for each data sample. This appears to be happening
with the posteriors of the hierarchical REC model estimated in Nimble.
And, as the size of the sample (i.e., \texttt{J}, referring to the
number of probable count sequences analyzed) increases, the variance of
the posterior densities decreases, as shown in the plots above. A bias
is being introduced because a single \(\beta\) parameter is being used
to estimate the likelihoods of all \texttt{J} probable regression models
simultaneously, which results in the estimates for \(\beta_j\) being
pulled together. This convergence, in turn, leads to a lower variance in
the upper level hyperparameter for \(\beta\)'s mean. Any lingering bias
in the underlying event sample, though, will cause that mean estimate to
deviate from the true underlying value. Subsequently, the hyperparameter
estimate for the target value will be a slightly biased, but
high-precision estimate.

To see the phenomenon more clearly, we can run an additional analysis
that involves a larger sample of dates. Whereas before a sample of 1000
dates was drawn from the simulated process, we can instead draw 5000
dates and re-run some of the analyses from above. The larger sample will
reduce the effect of sampling bias, and we should see an improvement in
the hierarchical model's estimate of the target value (\(\beta\)). It
will, however, likely still have a bias (from sampling and perhaps the
effect of chronological uncertainty on count sequences) and a lower
variance than either those involving a smaller sample of probable
sequences or a similar analysis conducted using the alternative sampling
framework implemented with \texttt{chronup}.

First, draw more samples from the simulation process,

\begin{Shaded}
\begin{Highlighting}[]
\NormalTok{nevents <-}\StringTok{ }\DecValTok{5000}
\NormalTok{nsamples <-}\StringTok{ }\DecValTok{200000}
\NormalTok{tau <-}\StringTok{ }\DecValTok{1000}
\NormalTok{beta <-}\StringTok{ }\FloatTok{0.004}
\NormalTok{x <-}\StringTok{ }\DecValTok{0}\OperatorTok{:}\NormalTok{(tau }\OperatorTok{-}\StringTok{ }\DecValTok{1}\NormalTok{)}
\NormalTok{lambda <-}\StringTok{ }\KeywordTok{exp}\NormalTok{(beta }\OperatorTok{*}\StringTok{ }\NormalTok{x)}
\NormalTok{start <-}\StringTok{ }\DecValTok{5000}
\NormalTok{end <-}\StringTok{ }\NormalTok{(start }\OperatorTok{-}\StringTok{ }\NormalTok{tau) }\OperatorTok{+}\StringTok{ }\DecValTok{1}
\NormalTok{times <-}\StringTok{ }\NormalTok{start}\OperatorTok{:}\NormalTok{end}
\end{Highlighting}
\end{Shaded}

\begin{Shaded}
\begin{Highlighting}[]
\NormalTok{simdata3 <-}\StringTok{ }\NormalTok{chronup}\OperatorTok{::}\KeywordTok{simulate_event_counts}\NormalTok{(}\DataTypeTok{process =}\NormalTok{ lambda,}
                                            \DataTypeTok{times =}\NormalTok{ times,}
                                            \DataTypeTok{nevents =}\NormalTok{ nevents,}
                                            \DataTypeTok{nsamples =}\NormalTok{ nsamples,}
                                            \DataTypeTok{binning_resolution =} \DecValTok{-10}\NormalTok{,}
                                            \DataTypeTok{bigmatrix =}\NormalTok{ datapath,}
                                            \DataTypeTok{bigfileprefix =} \StringTok{"count_big"}\NormalTok{)}
\KeywordTok{save}\NormalTok{(simdata3,}
    \DataTypeTok{file =} \KeywordTok{paste}\NormalTok{(datapath,}
                \StringTok{"simdata3.RData"}\NormalTok{,}
                \DataTypeTok{sep =} \StringTok{""}\NormalTok{))}
\end{Highlighting}
\end{Shaded}

Next, setup and run a Nimble model as before,

\begin{Shaded}
\begin{Highlighting}[]
\KeywordTok{library}\NormalTok{(nimble)}

\KeywordTok{load}\NormalTok{(}\KeywordTok{paste}\NormalTok{(datapath,}
        \StringTok{"simdata3.RData"}\NormalTok{,}
        \DataTypeTok{sep =} \StringTok{""}\NormalTok{))}

\NormalTok{rece <-}\StringTok{ }\NormalTok{bigmemory}\OperatorTok{::}\KeywordTok{attach.big.matrix}\NormalTok{(simdata3}\OperatorTok{$}\NormalTok{count_ensemble)}
\NormalTok{X <-}\StringTok{ }\NormalTok{bigmemory}\OperatorTok{::}\KeywordTok{attach.big.matrix}\NormalTok{(}\KeywordTok{paste}\NormalTok{(datapath,}
                                \StringTok{"X_mat_desc"}\NormalTok{,}
                                \DataTypeTok{sep =} \StringTok{""}\NormalTok{))}

\NormalTok{nbCode <-}\StringTok{ }\KeywordTok{nimbleCode}\NormalTok{(\{}
\NormalTok{   ###top-level regression}
\NormalTok{   B0 }\OperatorTok{~}\StringTok{ }\KeywordTok{dnorm}\NormalTok{(}\DecValTok{0}\NormalTok{, }\DecValTok{100}\NormalTok{)}
\NormalTok{   B1 }\OperatorTok{~}\StringTok{ }\KeywordTok{dnorm}\NormalTok{(}\DecValTok{0}\NormalTok{, }\DecValTok{100}\NormalTok{)}
\NormalTok{   sigB0 }\OperatorTok{~}\StringTok{ }\KeywordTok{dunif}\NormalTok{(}\FloatTok{1e-10}\NormalTok{, }\DecValTok{100}\NormalTok{)}
\NormalTok{   sigB1 }\OperatorTok{~}\StringTok{ }\KeywordTok{dunif}\NormalTok{(}\FloatTok{1e-10}\NormalTok{, }\DecValTok{100}\NormalTok{)}
   \ControlFlowTok{for}\NormalTok{ (j }\ControlFlowTok{in} \DecValTok{1}\OperatorTok{:}\NormalTok{J) \{}
\NormalTok{      ###low-level regression}
\NormalTok{      b0[j] }\OperatorTok{~}\StringTok{ }\KeywordTok{dnorm}\NormalTok{(}\DataTypeTok{mean =}\NormalTok{ B0, }\DataTypeTok{sd =}\NormalTok{ sigB0)}
\NormalTok{      b1[j] }\OperatorTok{~}\StringTok{ }\KeywordTok{dnorm}\NormalTok{(}\DataTypeTok{mean =}\NormalTok{ B1, }\DataTypeTok{sd =}\NormalTok{ sigB1)}
      \ControlFlowTok{for}\NormalTok{ (n }\ControlFlowTok{in} \DecValTok{1}\OperatorTok{:}\NormalTok{N)\{}
\NormalTok{        p[n, j] }\OperatorTok{~}\StringTok{ }\KeywordTok{dunif}\NormalTok{(}\FloatTok{1e-10}\NormalTok{, }\DecValTok{1}\NormalTok{)}
\NormalTok{        r[n, j] <-}\StringTok{ }\KeywordTok{exp}\NormalTok{(b0[j] }\OperatorTok{+}\StringTok{ }\NormalTok{X[n, j] }\OperatorTok{*}\StringTok{ }\NormalTok{b1[j])}
\NormalTok{        Y[n, j] }\OperatorTok{~}\StringTok{ }\KeywordTok{dnegbin}\NormalTok{(}\DataTypeTok{size =}\NormalTok{ r[n, j], }\DataTypeTok{prob =}\NormalTok{ p[n, j])}
\NormalTok{      \}}
\NormalTok{   \}}
\NormalTok{\})}

\NormalTok{span_index <-}\StringTok{ }\KeywordTok{which}\NormalTok{(simdata2}\OperatorTok{$}\NormalTok{new_timestamps }\OperatorTok{<=}\StringTok{ }\NormalTok{start }\OperatorTok{&}
\StringTok{                    }\NormalTok{simdata2}\OperatorTok{$}\NormalTok{new_timestamps }\OperatorTok{>=}\StringTok{ }\NormalTok{end)}
\NormalTok{n <-}\StringTok{ }\KeywordTok{length}\NormalTok{(span_index)}

\NormalTok{J =}\StringTok{ }\DecValTok{5}

\NormalTok{y_sample <-}\StringTok{ }\KeywordTok{sample}\NormalTok{(}\DecValTok{1}\OperatorTok{:}\KeywordTok{dim}\NormalTok{(rece)[}\DecValTok{2}\NormalTok{],}
                \DataTypeTok{size =}\NormalTok{ J,}
                \DataTypeTok{replace =}\NormalTok{ F)}
\NormalTok{x_sample <-}\StringTok{ }\KeywordTok{sample}\NormalTok{(}\KeywordTok{seq}\NormalTok{(}\DecValTok{2}\NormalTok{, }\KeywordTok{dim}\NormalTok{(rece)[}\DecValTok{2}\NormalTok{], }\DecValTok{2}\NormalTok{),}
                \DataTypeTok{size =}\NormalTok{ J,}
                \DataTypeTok{replace =}\NormalTok{ F)}

\NormalTok{nbData <-}\StringTok{ }\KeywordTok{list}\NormalTok{(}\DataTypeTok{Y =}\NormalTok{ rece[span_index, y_sample],}
                \DataTypeTok{X =}\NormalTok{ X[span_index, x_sample])}

\NormalTok{nbConsts <-}\StringTok{ }\KeywordTok{list}\NormalTok{(}\DataTypeTok{N =}\NormalTok{ n,}
                \DataTypeTok{J =}\NormalTok{ J)}

\NormalTok{nbInits <-}\StringTok{ }\KeywordTok{list}\NormalTok{(}\DataTypeTok{B0 =} \DecValTok{0}\NormalTok{,}
                \DataTypeTok{B1 =} \DecValTok{0}\NormalTok{,}
                \DataTypeTok{b0 =} \KeywordTok{rep}\NormalTok{(}\DecValTok{0}\NormalTok{, J),}
                \DataTypeTok{b1 =} \KeywordTok{rep}\NormalTok{(}\DecValTok{0}\NormalTok{, J),}
                \DataTypeTok{sigB0 =} \FloatTok{0.0001}\NormalTok{,}
                \DataTypeTok{sigB1 =} \FloatTok{0.0001}\NormalTok{)}

\NormalTok{nbModel <-}\StringTok{ }\KeywordTok{nimbleModel}\NormalTok{(}\DataTypeTok{code =}\NormalTok{ nbCode,}
                        \DataTypeTok{data =}\NormalTok{ nbData,}
                        \DataTypeTok{inits =}\NormalTok{ nbInits,}
                        \DataTypeTok{constants =}\NormalTok{ nbConsts)}

\CommentTok{#compile nimble model to C++ code—much faster runtime}
\NormalTok{C_nbModel <-}\StringTok{ }\KeywordTok{compileNimble}\NormalTok{(nbModel, }\DataTypeTok{showCompilerOutput =} \OtherTok{FALSE}\NormalTok{)}

\CommentTok{#configure the MCMC}
\NormalTok{nbModel_conf <-}\StringTok{ }\KeywordTok{configureMCMC}\NormalTok{(nbModel)}

\NormalTok{nbModel_conf}\OperatorTok{$}\NormalTok{monitors <-}\StringTok{ }\KeywordTok{c}\NormalTok{(}\StringTok{"B0"}\NormalTok{, }\StringTok{"B1"}\NormalTok{, }\StringTok{"sigB0"}\NormalTok{, }\StringTok{"sigB1"}\NormalTok{)}
\NormalTok{nbModel_conf}\OperatorTok{$}\KeywordTok{addMonitors2}\NormalTok{(}\KeywordTok{c}\NormalTok{(}\StringTok{"b0"}\NormalTok{, }\StringTok{"b1"}\NormalTok{))}

\CommentTok{#samplers}
\NormalTok{nbModel_conf}\OperatorTok{$}\KeywordTok{removeSamplers}\NormalTok{(}\KeywordTok{c}\NormalTok{(}\StringTok{"B0"}\NormalTok{, }\StringTok{"B1"}\NormalTok{, }\StringTok{"sigB0"}\NormalTok{, }\StringTok{"sigB1"}\NormalTok{, }\StringTok{"b0"}\NormalTok{, }\StringTok{"b1"}\NormalTok{))}
\NormalTok{nbModel_conf}\OperatorTok{$}\KeywordTok{addSampler}\NormalTok{(}\DataTypeTok{target =} \KeywordTok{c}\NormalTok{(}\StringTok{"B0"}\NormalTok{, }\StringTok{"B1"}\NormalTok{, }\StringTok{"sigB0"}\NormalTok{, }\StringTok{"sigB1"}\NormalTok{),}
                        \DataTypeTok{type =} \StringTok{"AF_slice"}\NormalTok{)}
\ControlFlowTok{for}\NormalTok{(j }\ControlFlowTok{in} \DecValTok{1}\OperatorTok{:}\NormalTok{J)\{}
\NormalTok{   nbModel_conf}\OperatorTok{$}\KeywordTok{addSampler}\NormalTok{(}\DataTypeTok{target =} \KeywordTok{c}\NormalTok{(}\KeywordTok{paste}\NormalTok{(}\StringTok{"b0["}\NormalTok{, j, }\StringTok{"]"}\NormalTok{, }\DataTypeTok{sep =} \StringTok{""}\NormalTok{),}
                                    \KeywordTok{paste}\NormalTok{(}\StringTok{"b1["}\NormalTok{, j, }\StringTok{"]"}\NormalTok{, }\DataTypeTok{sep =} \StringTok{""}\NormalTok{)),}
                        \DataTypeTok{type =} \StringTok{"AF_slice"}\NormalTok{)}
\NormalTok{\}}

\CommentTok{#thinning to conserve memory when the samples are saved below}
\NormalTok{thinning <-}\StringTok{ }\DecValTok{10}
\NormalTok{nbModel_conf}\OperatorTok{$}\KeywordTok{setThin}\NormalTok{(thinning)}
\NormalTok{nbModel_conf}\OperatorTok{$}\KeywordTok{setThin2}\NormalTok{(thinning)}

\KeywordTok{set.seed}\NormalTok{(}\DecValTok{1}\NormalTok{)}

\CommentTok{#build MCMC}
\NormalTok{nbModelMCMC <-}\StringTok{ }\KeywordTok{buildMCMC}\NormalTok{(nbModel_conf)}

\CommentTok{#compile MCMC to C++—much faster}
\NormalTok{C_nbModelMCMC <-}\StringTok{ }\KeywordTok{compileNimble}\NormalTok{(nbModelMCMC, }\DataTypeTok{project =}\NormalTok{ nbModel)}

\NormalTok{niter <-}\StringTok{ }\DecValTok{100000}

\CommentTok{#save the MCMC chain (monitored variables) as a matrix}
\NormalTok{nimble_samples_big <-}\StringTok{ }\KeywordTok{runMCMC}\NormalTok{(C_nbModelMCMC, }\DataTypeTok{niter =}\NormalTok{ niter)}

\CommentTok{#save samples}
\KeywordTok{save}\NormalTok{(nimble_samples_big,}
        \DataTypeTok{file =} \KeywordTok{paste}\NormalTok{(datapath,}
                    \StringTok{"mcmc_samples_nimble_big.RData"}\NormalTok{,}
                    \DataTypeTok{sep =} \StringTok{""}\NormalTok{))}
\end{Highlighting}
\end{Shaded}

Next, run the \texttt{chronup} analysis on the same large-sample
dataset,

\begin{Shaded}
\begin{Highlighting}[]
\KeywordTok{load}\NormalTok{(}\KeywordTok{paste}\NormalTok{(datapath,}
        \StringTok{"simdata3.RData"}\NormalTok{,}
        \DataTypeTok{sep =} \StringTok{""}\NormalTok{))}

\NormalTok{rece <-}\StringTok{ }\NormalTok{bigmemory}\OperatorTok{::}\KeywordTok{attach.big.matrix}\NormalTok{(simdata3}\OperatorTok{$}\NormalTok{count_ensemble)}
\NormalTok{X <-}\StringTok{ }\NormalTok{bigmemory}\OperatorTok{::}\KeywordTok{attach.big.matrix}\NormalTok{(}\KeywordTok{paste}\NormalTok{(datapath,}
                                \StringTok{"X_mat_desc"}\NormalTok{,}
                                \DataTypeTok{sep =} \StringTok{""}\NormalTok{))}

\NormalTok{nsamples <-}\StringTok{ }\KeywordTok{dim}\NormalTok{(rece)[}\DecValTok{2}\NormalTok{]}

\NormalTok{span_index <-}\StringTok{ }\KeywordTok{which}\NormalTok{(simdata3}\OperatorTok{$}\NormalTok{new_timestamps }\OperatorTok{<=}\StringTok{ }\NormalTok{start }\OperatorTok{&}
\StringTok{                    }\NormalTok{simdata3}\OperatorTok{$}\NormalTok{new_timestamps }\OperatorTok{>=}\StringTok{ }\NormalTok{end)}

\NormalTok{N <-}\StringTok{ }\KeywordTok{length}\NormalTok{(span_index)}

\NormalTok{startvals <-}\StringTok{ }\KeywordTok{c}\NormalTok{(}\DecValTok{0}\NormalTok{, }\DecValTok{0}\NormalTok{, }\KeywordTok{rep}\NormalTok{(}\FloatTok{0.1}\NormalTok{, N))}
\NormalTok{startscales <-}\StringTok{ }\KeywordTok{c}\NormalTok{(}\FloatTok{0.1}\NormalTok{, }\FloatTok{0.0002}\NormalTok{, }\KeywordTok{rep}\NormalTok{(}\FloatTok{0.5}\NormalTok{, N))}

\NormalTok{adapt_limit <-}\StringTok{ }\KeywordTok{floor}\NormalTok{(nsamples }\OperatorTok{*}\StringTok{ }\FloatTok{0.25}\NormalTok{)}

\NormalTok{adapt_sample_y <-}\StringTok{ }\DecValTok{1}\OperatorTok{:}\NormalTok{adapt_limit}
\NormalTok{adapt_sample_x <-}\StringTok{ }\DecValTok{1}\OperatorTok{:}\NormalTok{(adapt_limit }\OperatorTok{*}\StringTok{ }\DecValTok{2}\NormalTok{)}

\NormalTok{mcmc_samples_adapt <-}\StringTok{ }\NormalTok{chronup}\OperatorTok{::}\KeywordTok{regress}\NormalTok{(}\DataTypeTok{Y =}\NormalTok{ rece[span_index, adapt_sample_y],}
                                        \DataTypeTok{X =}\NormalTok{ X[span_index, adapt_sample_x],}
                                        \DataTypeTok{model =} \StringTok{"nb"}\NormalTok{,}
                                        \DataTypeTok{startvals =}\NormalTok{ startvals,}
                                        \DataTypeTok{scales =}\NormalTok{ startscales,}
                                        \DataTypeTok{adapt =}\NormalTok{ T)}

\NormalTok{burnin <-}\StringTok{ }\KeywordTok{floor}\NormalTok{(}\KeywordTok{dim}\NormalTok{(mcmc_samples_adapt}\OperatorTok{$}\NormalTok{samples)[}\DecValTok{1}\NormalTok{] }\OperatorTok{*}\StringTok{ }\FloatTok{0.2}\NormalTok{)}
\NormalTok{indeces <-}\StringTok{ }\KeywordTok{seq}\NormalTok{(burnin, }\KeywordTok{dim}\NormalTok{(mcmc_samples_adapt}\OperatorTok{$}\NormalTok{samples)[}\DecValTok{1}\NormalTok{], }\DecValTok{1}\NormalTok{)}
\NormalTok{adapt_startvals <-}\StringTok{ }\KeywordTok{colMeans}\NormalTok{(mcmc_samples_adapt}\OperatorTok{$}\NormalTok{samples[indeces, ])}

\NormalTok{burnin <-}\StringTok{ }\KeywordTok{floor}\NormalTok{(}\KeywordTok{dim}\NormalTok{(mcmc_samples_adapt}\OperatorTok{$}\NormalTok{scales)[}\DecValTok{1}\NormalTok{] }\OperatorTok{*}\StringTok{ }\FloatTok{0.2}\NormalTok{)}
\NormalTok{indeces <-}\StringTok{ }\KeywordTok{seq}\NormalTok{(burnin, }\KeywordTok{dim}\NormalTok{(mcmc_samples_adapt}\OperatorTok{$}\NormalTok{scales)[}\DecValTok{1}\NormalTok{], }\DecValTok{1}\NormalTok{)}
\NormalTok{adapt_startscales <-}\StringTok{ }\KeywordTok{colMeans}\NormalTok{(mcmc_samples_adapt}\OperatorTok{$}\NormalTok{scales[indeces, ])}

\NormalTok{chronup_samples_big <-}\StringTok{ }\NormalTok{chronup}\OperatorTok{::}\KeywordTok{regress}\NormalTok{(}\DataTypeTok{Y =}\NormalTok{ rece[span_index, ],}
                                \DataTypeTok{X =}\NormalTok{ X[span_index, ],}
                                \DataTypeTok{model =} \StringTok{"nb"}\NormalTok{,}
                                \DataTypeTok{startvals =}\NormalTok{ adapt_startvals,}
                                \DataTypeTok{scales =}\NormalTok{ adapt_startscales,}
                                \DataTypeTok{adapt =}\NormalTok{ F)}
\KeywordTok{save}\NormalTok{(chronup_samples_big,}
        \DataTypeTok{file =} \KeywordTok{paste}\NormalTok{(datapath,}
                    \StringTok{"mcmc_samples_chronup_big.RData"}\NormalTok{,}
                    \DataTypeTok{sep =} \StringTok{""}\NormalTok{))}
\end{Highlighting}
\end{Shaded}

Then, like last time, plot the results,

\begin{Shaded}
\begin{Highlighting}[]
\KeywordTok{load}\NormalTok{(}\KeywordTok{paste}\NormalTok{(datapath,}
            \StringTok{"mcmc_samples_nimble_5_big.RData"}\NormalTok{,}
            \DataTypeTok{sep =} \StringTok{""}\NormalTok{))}
\KeywordTok{load}\NormalTok{(}\KeywordTok{paste}\NormalTok{(datapath,}
            \StringTok{"mcmc_samples_nimble_10_big.RData"}\NormalTok{,}
            \DataTypeTok{sep =} \StringTok{""}\NormalTok{))}
\KeywordTok{load}\NormalTok{(}\KeywordTok{paste}\NormalTok{(datapath,}
            \StringTok{"mcmc_samples_nimble_50_big.RData"}\NormalTok{,}
            \DataTypeTok{sep =} \StringTok{""}\NormalTok{))}
\KeywordTok{load}\NormalTok{(}\KeywordTok{paste}\NormalTok{(datapath,}
            \StringTok{"mcmc_samples_nimble_100_big.RData"}\NormalTok{,}
            \DataTypeTok{sep =} \StringTok{""}\NormalTok{))}
\KeywordTok{load}\NormalTok{(}\KeywordTok{paste}\NormalTok{(datapath,}
            \StringTok{"mcmc_samples_chronup_big.RData"}\NormalTok{,}
            \DataTypeTok{sep =} \StringTok{""}\NormalTok{))}

\NormalTok{n_hier <-}\StringTok{ }\KeywordTok{dim}\NormalTok{(nimble_samples_}\DecValTok{5}\NormalTok{_big}\OperatorTok{$}\NormalTok{samples)[}\DecValTok{1}\NormalTok{]}
\NormalTok{n_alt <-}\StringTok{ }\KeywordTok{dim}\NormalTok{(chronup_samples_big)[}\DecValTok{1}\NormalTok{]}

\NormalTok{burnin_hier <-}\StringTok{ }\KeywordTok{floor}\NormalTok{(n_hier }\OperatorTok{*}\StringTok{ }\FloatTok{0.1}\NormalTok{)}
\NormalTok{burnin_alt <-}\StringTok{ }\KeywordTok{floor}\NormalTok{(n_alt }\OperatorTok{*}\StringTok{ }\FloatTok{0.1}\NormalTok{)}

\NormalTok{iteration_hier <-}\StringTok{ }\NormalTok{(burnin_hier }\OperatorTok{+}\StringTok{ }\DecValTok{1}\NormalTok{)}\OperatorTok{:}\NormalTok{n_hier}
\NormalTok{iteration_hier_all <-}\StringTok{ }\KeywordTok{rep}\NormalTok{(iteration_hier, }\DecValTok{4}\NormalTok{)}
\NormalTok{iteration_alt <-}\StringTok{ }\NormalTok{(burnin_alt }\OperatorTok{+}\StringTok{ }\DecValTok{1}\NormalTok{)}\OperatorTok{:}\NormalTok{n_alt}

\NormalTok{n_retained_hier <-}\StringTok{ }\KeywordTok{length}\NormalTok{(iteration_hier)}
\NormalTok{n_retained_alt <-}\StringTok{ }\KeywordTok{length}\NormalTok{(iteration_alt)}

\NormalTok{plot_samples_hier_}\DecValTok{5}\NormalTok{ <-}\StringTok{ }\NormalTok{nimble_samples_}\DecValTok{5}\NormalTok{_big}\OperatorTok{$}\NormalTok{samples[iteration_hier, }\DecValTok{2}\NormalTok{]}
\NormalTok{plot_samples_hier_}\DecValTok{10}\NormalTok{ <-}\StringTok{ }\NormalTok{nimble_samples_}\DecValTok{10}\NormalTok{_big}\OperatorTok{$}\NormalTok{samples[iteration_hier, }\DecValTok{2}\NormalTok{]}
\NormalTok{plot_samples_hier_}\DecValTok{50}\NormalTok{ <-}\StringTok{ }\NormalTok{nimble_samples_}\DecValTok{50}\NormalTok{_big}\OperatorTok{$}\NormalTok{samples[iteration_hier, }\DecValTok{2}\NormalTok{]}
\NormalTok{plot_samples_hier_}\DecValTok{100}\NormalTok{ <-}\StringTok{ }\NormalTok{nimble_samples_}\DecValTok{100}\NormalTok{_big}\OperatorTok{$}\NormalTok{samples[iteration_hier, }\DecValTok{2}\NormalTok{]}
\NormalTok{plot_samples_hier <-}\StringTok{ }\KeywordTok{c}\NormalTok{(plot_samples_hier_}\DecValTok{5}\NormalTok{,}
\NormalTok{                    plot_samples_hier_}\DecValTok{10}\NormalTok{,}
\NormalTok{                    plot_samples_hier_}\DecValTok{50}\NormalTok{,}
\NormalTok{                    plot_samples_hier_}\DecValTok{100}\NormalTok{)}
\NormalTok{plot_samples_alt <-}\StringTok{ }\NormalTok{chronup_samples_big[iteration_alt, }\DecValTok{2}\NormalTok{]}

\NormalTok{model_name_hier <-}\StringTok{ }\KeywordTok{rep}\NormalTok{(}\StringTok{"hier"}\NormalTok{, n_retained_hier }\OperatorTok{*}\StringTok{ }\DecValTok{4}\NormalTok{)}
\NormalTok{model_name_alt <-}\StringTok{ }\KeywordTok{rep}\NormalTok{(}\StringTok{"alt"}\NormalTok{, n_retained_alt)}

\NormalTok{J_hier <-}\StringTok{ }\KeywordTok{rep}\NormalTok{(}\KeywordTok{c}\NormalTok{(}\StringTok{"5"}\NormalTok{, }\StringTok{"10"}\NormalTok{, }\StringTok{"50"}\NormalTok{, }\StringTok{"100"}\NormalTok{), }\DataTypeTok{each =}\NormalTok{ n_retained_hier)}
\NormalTok{J_alt <-}\StringTok{ }\KeywordTok{rep}\NormalTok{(}\StringTok{"200000"}\NormalTok{, n_retained_alt)}

\NormalTok{J_both <-}\StringTok{ }\KeywordTok{factor}\NormalTok{(}\KeywordTok{c}\NormalTok{(J_hier, J_alt), }\DataTypeTok{levels =} \KeywordTok{c}\NormalTok{(}\StringTok{"5"}\NormalTok{, }\StringTok{"10"}\NormalTok{, }\StringTok{"50"}\NormalTok{, }\StringTok{"100"}\NormalTok{, }\StringTok{"200000"}\NormalTok{))}

\NormalTok{mcmc_samples_both <-}\StringTok{ }\KeywordTok{data.frame}\NormalTok{(}\DataTypeTok{Iteration =} \KeywordTok{c}\NormalTok{(iteration_hier_all,}
\NormalTok{                                            iteration_alt),}
                            \DataTypeTok{Model =} \KeywordTok{c}\NormalTok{(model_name_hier,}
\NormalTok{                                    model_name_alt),}
                            \DataTypeTok{J =}\NormalTok{ J_both,}
                            \DataTypeTok{Beta =} \KeywordTok{c}\NormalTok{(plot_samples_hier,}
\NormalTok{                                    plot_samples_alt))}

\KeywordTok{ggplot}\NormalTok{(}\DataTypeTok{data =}\NormalTok{ mcmc_samples_both) }\OperatorTok{+}
\StringTok{    }\KeywordTok{geom_vline}\NormalTok{(}\DataTypeTok{xintercept =}\NormalTok{ beta,}
                \DataTypeTok{linetype =} \DecValTok{2}\NormalTok{,}
                \DataTypeTok{alpha =} \FloatTok{0.8}\NormalTok{) }\OperatorTok{+}
\StringTok{    }\KeywordTok{geom_density}\NormalTok{(}\DataTypeTok{mapping =} \KeywordTok{aes}\NormalTok{(}\DataTypeTok{y =}\NormalTok{ ..scaled.., }\DataTypeTok{x =}\NormalTok{ Beta),}
                \DataTypeTok{fill =} \StringTok{"steelblue"}\NormalTok{,}
                \DataTypeTok{alpha =} \FloatTok{0.5}\NormalTok{) }\OperatorTok{+}
\StringTok{    }\KeywordTok{facet_wrap}\NormalTok{(}\OperatorTok{~}\StringTok{ }\NormalTok{J, }\DataTypeTok{ncol =} \DecValTok{2}\NormalTok{, }\DataTypeTok{scales =} \StringTok{"free"}\NormalTok{) }\OperatorTok{+}
\StringTok{    }\KeywordTok{theme_minimal}\NormalTok{()}
\end{Highlighting}
\end{Shaded}

\includegraphics{replication_files/figure-latex/plot-mcmc-densities-big-1.pdf}

As before, to make comparisons easier, we can re-produce the same plot,
but this time limit the x-axis to a region around the target value for
\(\beta\) and align the plot axes:

\begin{Shaded}
\begin{Highlighting}[]
\KeywordTok{ggplot}\NormalTok{(}\DataTypeTok{data =}\NormalTok{ mcmc_samples_both) }\OperatorTok{+}
\StringTok{    }\KeywordTok{geom_vline}\NormalTok{(}\DataTypeTok{xintercept =}\NormalTok{ beta,}
                \DataTypeTok{linetype =} \DecValTok{2}\NormalTok{,}
                \DataTypeTok{alpha =} \FloatTok{0.8}\NormalTok{) }\OperatorTok{+}
\StringTok{    }\KeywordTok{geom_density}\NormalTok{(}\DataTypeTok{mapping =} \KeywordTok{aes}\NormalTok{(}\DataTypeTok{y =}\NormalTok{ ..scaled.., }\DataTypeTok{x =}\NormalTok{ Beta),}
                \DataTypeTok{fill =} \StringTok{"steelblue"}\NormalTok{,}
                \DataTypeTok{alpha =} \FloatTok{0.5}\NormalTok{) }\OperatorTok{+}
\StringTok{    }\KeywordTok{facet_wrap}\NormalTok{(}\OperatorTok{~}\StringTok{ }\NormalTok{J, }\DataTypeTok{ncol =} \DecValTok{2}\NormalTok{, }\DataTypeTok{scales =} \StringTok{"free"}\NormalTok{) }\OperatorTok{+}
\StringTok{    }\KeywordTok{scale_x_continuous}\NormalTok{(}\DataTypeTok{limits =} \KeywordTok{c}\NormalTok{(}\FloatTok{0.001}\NormalTok{, }\FloatTok{0.007}\NormalTok{)) }\OperatorTok{+}
\StringTok{    }\KeywordTok{theme_minimal}\NormalTok{()}
\end{Highlighting}
\end{Shaded}

\includegraphics{replication_files/figure-latex/plot-mcmc-densities2-big-1.pdf}

Lastly, we should also inspect the mcmc trace plots for each
model/estimate, and then examine the Geweke statistics for the main
parameter,

\begin{Shaded}
\begin{Highlighting}[]
\KeywordTok{ggplot}\NormalTok{(}\DataTypeTok{data =}\NormalTok{ mcmc_samples_both) }\OperatorTok{+}
\StringTok{    }\KeywordTok{geom_hline}\NormalTok{(}\DataTypeTok{yintercept =}\NormalTok{ beta,}
                \DataTypeTok{linetype =} \DecValTok{2}\NormalTok{,}
                \DataTypeTok{alpha =} \FloatTok{0.8}\NormalTok{) }\OperatorTok{+}
\StringTok{    }\KeywordTok{geom_path}\NormalTok{(}\DataTypeTok{mapping =} \KeywordTok{aes}\NormalTok{(}\DataTypeTok{y =}\NormalTok{ Beta, }\DataTypeTok{x =}\NormalTok{ Iteration),}
                \DataTypeTok{alpha =} \FloatTok{0.85}\NormalTok{) }\OperatorTok{+}
\StringTok{    }\KeywordTok{facet_wrap}\NormalTok{(}\OperatorTok{~}\StringTok{ }\NormalTok{J, }\DataTypeTok{ncol =} \DecValTok{2}\NormalTok{, }\DataTypeTok{scales =} \StringTok{"free"}\NormalTok{) }\OperatorTok{+}
\StringTok{    }\KeywordTok{theme_minimal}\NormalTok{()}
\end{Highlighting}
\end{Shaded}

\includegraphics{replication_files/figure-latex/plot-mcmc-trace-big-1.pdf}

\begin{Shaded}
\begin{Highlighting}[]
\KeywordTok{geweke.diag}\NormalTok{(}\KeywordTok{cbind}\NormalTok{(plot_samples_hier_}\DecValTok{5}\NormalTok{,}
\NormalTok{                plot_samples_hier_}\DecValTok{10}\NormalTok{,}
\NormalTok{                plot_samples_hier_}\DecValTok{50}\NormalTok{,}
\NormalTok{                plot_samples_hier_}\DecValTok{100}\NormalTok{,}
\NormalTok{                plot_samples_alt))}
\end{Highlighting}
\end{Shaded}

\begin{verbatim}
## Warning in cbind(plot_samples_hier_5, plot_samples_hier_10,
## plot_samples_hier_50, : number of rows of result is not a multiple of vector
## length (arg 1)
\end{verbatim}

\begin{verbatim}
## 
## Fraction in 1st window = 0.1
## Fraction in 2nd window = 0.5 
## 
##   plot_samples_hier_5  plot_samples_hier_10  plot_samples_hier_50 
##             -0.347998             -0.004298              0.037246 
## plot_samples_hier_100      plot_samples_alt 
##             -0.159882             -1.389440
\end{verbatim}

\end{document}
